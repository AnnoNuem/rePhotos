%% Generated by Sphinx.
\def\sphinxdocclass{report}
\documentclass[letterpaper,10pt,english]{sphinxmanual}
\ifdefined\pdfpxdimen
   \let\sphinxpxdimen\pdfpxdimen\else\newdimen\sphinxpxdimen
\fi \sphinxpxdimen=49336sp\relax

\usepackage[margin=1in,marginparwidth=0.5in]{geometry}
\usepackage[utf8]{inputenc}
\ifdefined\DeclareUnicodeCharacter
  \DeclareUnicodeCharacter{00A0}{\nobreakspace}
\fi
\usepackage{cmap}
\usepackage[T1]{fontenc}
\usepackage{amsmath,amssymb,amstext}
\usepackage{babel}
\usepackage{times}
\usepackage[Bjarne]{fncychap}
\usepackage{longtable}
\usepackage{sphinx}

\usepackage{multirow}
\usepackage{eqparbox}

% Include hyperref last.
\usepackage{hyperref}
% Fix anchor placement for figures with captions.
\usepackage{hypcap}% it must be loaded after hyperref.
% Set up styles of URL: it should be placed after hyperref.
\urlstyle{same}
\addto\captionsenglish{\renewcommand{\contentsname}{Contents:}}

\addto\captionsenglish{\renewcommand{\figurename}{Fig.}}
\addto\captionsenglish{\renewcommand{\tablename}{Table}}
\addto\captionsenglish{\renewcommand{\literalblockname}{Listing}}

\addto\extrasenglish{\def\pageautorefname{page}}

\setcounter{tocdepth}{1}



\title{rePhotosImageAlignment Documentation}
\date{Feb 17, 2017}
\release{1.0}
\author{Axel Schaffland}
\newcommand{\sphinxlogo}{}
\renewcommand{\releasename}{Release}
\makeindex

\begin{document}

\maketitle
\sphinxtableofcontents
\phantomsection\label{\detokenize{index::doc}}



\chapter{aaap\_re\_photo module}
\label{\detokenize{aaap_re_photo:aaap-re-photo-module}}\label{\detokenize{aaap_re_photo::doc}}\label{\detokenize{aaap_re_photo:welcome-to-rephotosimagealignment-s-documentation}}\phantomsection\label{\detokenize{aaap_re_photo:module-aaap_re_photo}}\index{aaap\_re\_photo (module)}
Mainscript for image alignment in the re.photos project.
Two images which names are given as parameters at program start are loaded and
scaled to the same size. In a two stage approach the user aligns the images. In
the first stage the user draws rectangles on either of the two images. A
probable corner point inside the rectangle meant by the user is automatically
selected as well as a corresponding point in the second image. Alternatively the
user can draw points directly. With four pointpairs a perspective transform is
performed aligning the images roughly. In the second stage the user draws lines
instead of point and as-affine-as-possible warping is used to tune the image
alignment. After that the images are cropped to the maximal possible size.
\index{init() (in module aaap\_re\_photo)}

\begin{fulllineitems}
\phantomsection\label{\detokenize{aaap_re_photo:aaap_re_photo.init}}\pysiglinewithargsret{\sphinxcode{aaap\_re\_photo.}\sphinxbfcode{init}}{}{}
Processes command line parameters, reads images, lines and points.
\begin{quote}\begin{description}
\item[{Returns}] \leavevmode
\begin{itemize}
\item {} 
\sphinxstylestrong{src\_img} (\sphinxstyleemphasis{ndarray}) -- Image which is warped.

\item {} 
\sphinxstylestrong{dst\_img} (\sphinxstyleemphasis{ndarray}) -- Image to which the src\_img is warped.

\item {} 
\sphinxstylestrong{src\_lines} (\sphinxstyleemphasis{list}) -- List of lines in src\_img read from line file. If line file not found or
command line parameter \sphinxstyleemphasis{--line\_file} not given an empty list is returned.

\item {} 
\sphinxstylestrong{dst\_lines} (\sphinxstyleemphasis{list}) -- List of lines in dst\_img read from line file. If line file not found or
command line parameter \sphinxstyleemphasis{--line\_file} not given an empty list is returned.

\item {} 
\sphinxstylestrong{src\_points} (\sphinxstyleemphasis{list}) -- List of points in src\_img read from point file. If point file not found or
command line parameter \sphinxstyleemphasis{--point\_file} not given an empty list is returned.

\item {} 
\sphinxstylestrong{dst\_points} (\sphinxstyleemphasis{list}) -- List of points in dst\_img read from point file. If point file not found or
command line parameter \sphinxstyleemphasis{-point\_file} not given an empty list is returned.

\item {} 
\sphinxstylestrong{args} (\sphinxstyleemphasis{Namespace}) -- Command line arguments.

\end{itemize}


\end{description}\end{quote}

\end{fulllineitems}

\index{ls() (in module aaap\_re\_photo)}

\begin{fulllineitems}
\phantomsection\label{\detokenize{aaap_re_photo:aaap_re_photo.ls}}\pysiglinewithargsret{\sphinxcode{aaap\_re\_photo.}\sphinxbfcode{ls}}{\emph{l}, \emph{sf}}{}
\end{fulllineitems}

\index{main() (in module aaap\_re\_photo)}

\begin{fulllineitems}
\phantomsection\label{\detokenize{aaap_re_photo:aaap_re_photo.main}}\pysiglinewithargsret{\sphinxcode{aaap\_re\_photo.}\sphinxbfcode{main}}{}{}
First function to be called. Initializes programm and switches stages.

\end{fulllineitems}

\index{onMouse\_stage\_one() (in module aaap\_re\_photo)}

\begin{fulllineitems}
\phantomsection\label{\detokenize{aaap_re_photo:aaap_re_photo.onMouse_stage_one}}\pysiglinewithargsret{\sphinxcode{aaap\_re\_photo.}\sphinxbfcode{onMouse\_stage\_one}}{}{}
Mousecallback function for first stage user input.
User input is drawn on resized images. Search for point and corresponding
point is done on original sized images.
Custom parameters are given as one tuple.
\begin{quote}\begin{description}
\item[{Parameters}] \leavevmode\begin{itemize}
\item {} 
\sphinxstyleliteralstrong{event} (\sphinxstyleliteralemphasis{int}) -- Mouse event.

\item {} 
\sphinxstyleliteralstrong{x} (\sphinxstyleliteralemphasis{int}) -- X-coordinate of mouse pointer.

\item {} 
\sphinxstyleliteralstrong{y} (\sphinxstyleliteralemphasis{int}) -- Y-coordinate of mouse pointer.

\item {} 
\sphinxstyleliteralstrong{flags} (\sphinxstyleliteralemphasis{int}) -- Type of mouse event.

\item {} 
\sphinxstyleliteralstrong{img\_d} (\sphinxstyleliteralemphasis{ndarray}) -- Resized image on which points are drawn.

\item {} 
\sphinxstyleliteralstrong{img\_d\_clean} (\sphinxstyleliteralemphasis{ndarray}) -- Resized image to reset img\_d if points are removed.

\item {} 
\sphinxstyleliteralstrong{img\_orig} (\sphinxstyleliteralemphasis{ndarray}) -- Unscaled image in which point is searched.

\item {} 
\sphinxstyleliteralstrong{scale} (\sphinxstyleliteralemphasis{float}) -- Scale of img\_d with respect to imig\_orig.

\item {} 
\sphinxstyleliteralstrong{points} (\sphinxstyleliteralemphasis{list}) -- List of points.

\item {} 
\sphinxstyleliteralstrong{win\_name} (\sphinxstyleliteralemphasis{string}) -- Name of window in which image is shown.

\item {} 
\sphinxstyleliteralstrong{color} (\sphinxstyleliteralemphasis{tuple}) -- Color of drawn points.

\item {} 
\sphinxstyleliteralstrong{img2\_d} (\sphinxstyleliteralemphasis{ndarray}) -- Resized image for corresponding point on which points are drawn.

\item {} 
\sphinxstyleliteralstrong{img2\_d\_clean} (\sphinxstyleliteralemphasis{ndarray}) -- Resized image for corresponding point to reset img\_d if points
are removed.

\item {} 
\sphinxstyleliteralstrong{img2\_orig} (\sphinxstyleliteralemphasis{ndarray}) -- Unscaled image for corresponding point in which corresponding
point is searched.

\item {} 
\sphinxstyleliteralstrong{scale2} (\sphinxstyleliteralemphasis{float}) -- Scale of img2\_d with respect to imig2\_orig.

\item {} 
\sphinxstyleliteralstrong{points2} (\sphinxstyleliteralemphasis{list}) -- List of corresponding points.

\item {} 
\sphinxstyleliteralstrong{win\_name2} (\sphinxstyleliteralemphasis{string}) -- Name of window in which image2 is shown.

\item {} 
\sphinxstyleliteralstrong{color2} (\sphinxstyleliteralemphasis{tuple}) -- Color of corresponding points.

\end{itemize}

\end{description}\end{quote}

\end{fulllineitems}

\index{onMouse\_stage\_two() (in module aaap\_re\_photo)}

\begin{fulllineitems}
\phantomsection\label{\detokenize{aaap_re_photo:aaap_re_photo.onMouse_stage_two}}\pysiglinewithargsret{\sphinxcode{aaap\_re\_photo.}\sphinxbfcode{onMouse\_stage\_two}}{}{}
Mousecallback function for second stage user input.
User input is drawn on resized images. Search for line and corresponding
line is done on original sized images.
Custom parameters are given as one tuple.
\begin{quote}\begin{description}
\item[{Parameters}] \leavevmode\begin{itemize}
\item {} 
\sphinxstyleliteralstrong{event} (\sphinxstyleliteralemphasis{int}) -- Mouse event.

\item {} 
\sphinxstyleliteralstrong{x} (\sphinxstyleliteralemphasis{int}) -- X-coordinate of mouse pointer.

\item {} 
\sphinxstyleliteralstrong{y} (\sphinxstyleliteralemphasis{int}) -- Y-coordinate of mouse pointer.

\item {} 
\sphinxstyleliteralstrong{flags} (\sphinxstyleliteralemphasis{int}) -- Type of mouse event.

\item {} 
\sphinxstyleliteralstrong{img\_d} (\sphinxstyleliteralemphasis{ndarray}) -- Resized image on which lines are drawn.

\item {} 
\sphinxstyleliteralstrong{img\_d\_clean} (\sphinxstyleliteralemphasis{ndarray}) -- Resized image to reset img\_d if lines are removed.

\item {} 
\sphinxstyleliteralstrong{img\_orig} (\sphinxstyleliteralemphasis{ndarray}) -- Unscaled image in which lines is searched.

\item {} 
\sphinxstyleliteralstrong{scale} (\sphinxstyleliteralemphasis{float}) -- Scale of img\_d with respect to imig\_orig.

\item {} 
\sphinxstyleliteralstrong{lines} (\sphinxstyleliteralemphasis{list}) -- List of lines.

\item {} 
\sphinxstyleliteralstrong{win\_name} (\sphinxstyleliteralemphasis{string}) -- Name of window in which image is shown.

\item {} 
\sphinxstyleliteralstrong{color} (\sphinxstyleliteralemphasis{tuple}) -- Color of drawn lines.

\item {} 
\sphinxstyleliteralstrong{img2\_d} (\sphinxstyleliteralemphasis{ndarray}) -- Resized image for corresponding lines on which lines are drawn.

\item {} 
\sphinxstyleliteralstrong{img2\_d\_clean} (\sphinxstyleliteralemphasis{ndarray}) -- Resized image for corresponding lines to reset img\_d if lines
are removed.

\item {} 
\sphinxstyleliteralstrong{img2\_orig} (\sphinxstyleliteralemphasis{ndarray}) -- Unscaled image for corresponding line in which corresponding
line is searched.

\item {} 
\sphinxstyleliteralstrong{scale2} (\sphinxstyleliteralemphasis{float}) -- Scale of img2\_d with respect to imig2\_orig.

\item {} 
\sphinxstyleliteralstrong{points2} (\sphinxstyleliteralemphasis{list}) -- List of corresponding lines.

\item {} 
\sphinxstyleliteralstrong{win\_name2} (\sphinxstyleliteralemphasis{string}) -- Name of window in which image2 is shown.

\item {} 
\sphinxstyleliteralstrong{color2} (\sphinxstyleliteralemphasis{tuple}) -- Color of corresponding lines.

\end{itemize}

\end{description}\end{quote}

\end{fulllineitems}

\index{ps() (in module aaap\_re\_photo)}

\begin{fulllineitems}
\phantomsection\label{\detokenize{aaap_re_photo:aaap_re_photo.ps}}\pysiglinewithargsret{\sphinxcode{aaap\_re\_photo.}\sphinxbfcode{ps}}{\emph{p}, \emph{sf}}{}
\end{fulllineitems}

\index{stage\_one() (in module aaap\_re\_photo)}

\begin{fulllineitems}
\phantomsection\label{\detokenize{aaap_re_photo:aaap_re_photo.stage_one}}\pysiglinewithargsret{\sphinxcode{aaap\_re\_photo.}\sphinxbfcode{stage\_one}}{\emph{src\_img}, \emph{dst\_img}, \emph{src\_points}, \emph{dst\_points}, \emph{src\_lines}, \emph{dst\_lines}, \emph{args}}{}
First stage. User drawn points are used for perspective alignment.
\begin{quote}\begin{description}
\item[{Parameters}] \leavevmode\begin{itemize}
\item {} 
\sphinxstyleliteralstrong{src\_img} (\sphinxstyleliteralemphasis{ndarray}) -- Image on which aaap-warping will be later performed.

\item {} 
\sphinxstyleliteralstrong{dst\_img} (\sphinxstyleliteralemphasis{ndarray}) -- Image which is only perspective transformed.

\item {} 
\sphinxstyleliteralstrong{src\_points} (\sphinxstyleliteralemphasis{list}) -- Points in src\_img.

\item {} 
\sphinxstyleliteralstrong{dst\_points} (\sphinxstyleliteralemphasis{list}) -- Points in dst\_img.

\item {} 
\sphinxstyleliteralstrong{src\_lines} (\sphinxstyleliteralemphasis{list}) -- Lines in src\_img.

\item {} 
\sphinxstyleliteralstrong{dst\_lines} (\sphinxstyleliteralemphasis{list}) -- Lines in dst\_img

\item {} 
\sphinxstyleliteralstrong{args} (\sphinxstyleliteralemphasis{Namespace}) -- Parameters.

\end{itemize}

\item[{Returns}] \leavevmode
\begin{itemize}
\item {} 
\sphinxstylestrong{src\_img} (\sphinxstyleemphasis{ndarray}) -- Perspective transformed src\_img.

\item {} 
\sphinxstylestrong{dst\_img} (\sphinxstyleemphasis{ndarray}) -- Perspective transformed dst\_img.

\item {} 
\sphinxstylestrong{src\_points} (\sphinxstyleemphasis{list}) -- Perspective transformed points in src\_img.

\item {} 
\sphinxstylestrong{dst\_points} (\sphinxstyleemphasis{list}) -- Perspective transformed points in dst\_img.

\item {} 
\sphinxstylestrong{src\_lines} (\sphinxstyleemphasis{list}) -- Perspective transformed lines in src\_img.

\item {} 
\sphinxstylestrong{dst\_lines} (\sphinxstyleemphasis{list}) -- Perspective transformed lines in dst\_img.

\item {} 
\sphinxstylestrong{src\_transform\_matrix} (\sphinxstyleemphasis{ndarray}) -- Perspective transform matrix of src image, lines and points.

\item {} 
\sphinxstylestrong{dst\_transform\_matrix} (\sphinxstyleemphasis{ndarray}) -- Perspective transform matrix of dst image, lines and points.

\item {} 
\sphinxstylestrong{stage\_one\_success} (\sphinxstyleemphasis{bool}) -- True if four point pairs where avaiable to perform perspective
transform of src and dst images, lines and points, else False.

\end{itemize}


\end{description}\end{quote}

\end{fulllineitems}

\index{stage\_two() (in module aaap\_re\_photo)}

\begin{fulllineitems}
\phantomsection\label{\detokenize{aaap_re_photo:aaap_re_photo.stage_two}}\pysiglinewithargsret{\sphinxcode{aaap\_re\_photo.}\sphinxbfcode{stage\_two}}{\emph{src\_img}, \emph{dst\_img}, \emph{src\_points}, \emph{dst\_points}, \emph{src\_lines}, \emph{dst\_lines}, \emph{src\_transform\_matrix}, \emph{dst\_transform\_matrix}, \emph{stage\_one\_success}, \emph{args}}{}
Second stage. User drawn lines are used for aaap-warping.
\begin{quote}\begin{description}
\item[{Parameters}] \leavevmode\begin{itemize}
\item {} 
\sphinxstyleliteralstrong{src\_img} (\sphinxstyleliteralemphasis{ndarray}) -- Image on which aaap-warping will be later performed.

\item {} 
\sphinxstyleliteralstrong{dst\_img} (\sphinxstyleliteralemphasis{ndarray}) -- Image to which src\_img shall be warped.

\item {} 
\sphinxstyleliteralstrong{src\_points} (\sphinxstyleliteralemphasis{list}) -- Points in src\_img.

\item {} 
\sphinxstyleliteralstrong{dst\_points} (\sphinxstyleliteralemphasis{list}) -- Points in dst\_img.

\item {} 
\sphinxstyleliteralstrong{src\_lines} (\sphinxstyleliteralemphasis{list}) -- Lines in src\_img.

\item {} 
\sphinxstyleliteralstrong{dst\_lines} (\sphinxstyleliteralemphasis{list}) -- Lines in dst\_img

\item {} 
\sphinxstyleliteralstrong{src\_transform\_matrix} (\sphinxstyleliteralemphasis{ndarray}) -- Perspective transform matrix of src image, lines and points
from first stage.

\item {} 
\sphinxstyleliteralstrong{dst\_transform\_matrix} (\sphinxstyleliteralemphasis{ndarray}) -- Perspective transform matrix of dst image, lines and points
from first stage.

\item {} 
\sphinxstyleliteralstrong{stage\_one\_success} (\sphinxstyleliteralemphasis{bool}) -- True if first stage was successful, else False.

\item {} 
\sphinxstyleliteralstrong{args} (\sphinxstyleliteralemphasis{Namespace}) -- Parameters.

\end{itemize}

\end{description}\end{quote}

\end{fulllineitems}



\chapter{image\_aaap module}
\label{\detokenize{image_aaap:image-aaap-module}}\label{\detokenize{image_aaap::doc}}\label{\detokenize{image_aaap:module-image_aaap}}\index{image\_aaap (module)}
Functions for As-Affine-As-Possible Warping as described in
`Generalized As-Similar-As-Possible Warping with
Applications in Digital Photography' by Chen and Gotsman.
\index{bilinear\_point\_in\_quad\_mesh() (in module image\_aaap)}

\begin{fulllineitems}
\phantomsection\label{\detokenize{image_aaap:image_aaap.bilinear_point_in_quad_mesh}}\pysiglinewithargsret{\sphinxcode{image\_aaap.}\sphinxbfcode{bilinear\_point\_in\_quad\_mesh}}{\emph{pts}, \emph{X}, \emph{P}, \emph{qmSize}}{}
Express points in a quad mesh as the convex combination of there
containing quads, using bilinear weights
A = bilinearPointInQuadMesh(pts, X, P, qmSize)
\begin{quote}\begin{description}
\item[{Parameters}] \leavevmode\begin{itemize}
\item {} 
\sphinxstyleliteralstrong{pts} (\sphinxstyleliteralemphasis{ndarray}) -- Points that are to be expressed as bilinear combinations of
the quadmesh vertices

\item {} 
\sphinxstyleliteralstrong{X} (\sphinxstyleliteralemphasis{ndarray}) -- The vertices of the quadmesh

\item {} 
\sphinxstyleliteralstrong{P} (\sphinxstyleliteralemphasis{ndarray}) -- The connectivy of the quadmesh

\item {} 
\sphinxstyleliteralstrong{qmSize} (\sphinxstyleliteralemphasis{tuple}) -- Size (rows/columns of quads) of the quadmesh, that is
consctructed to cover some image plane.

\end{itemize}

\item[{Returns}] \leavevmode
\sphinxstylestrong{Ascr} -- A matrix that gives the weights for the points as combinations
of the quadmesh vertices, i.e. A*X = pts.

\item[{Return type}] \leavevmode
csc\_matrix

\end{description}\end{quote}

\end{fulllineitems}

\index{build\_regular\_mesh() (in module image\_aaap)}

\begin{fulllineitems}
\phantomsection\label{\detokenize{image_aaap:image_aaap.build_regular_mesh}}\pysiglinewithargsret{\sphinxcode{image\_aaap.}\sphinxbfcode{build\_regular\_mesh}}{\emph{width}, \emph{height}, \emph{grid\_size}}{}
Creates quadratic meshgrid of given width, height and distance between
grid points.
\begin{quote}\begin{description}
\item[{Parameters}] \leavevmode\begin{itemize}
\item {} 
\sphinxstyleliteralstrong{width} (\sphinxstyleliteralemphasis{int}) -- Width of meshgrid.

\item {} 
\sphinxstyleliteralstrong{height} (\sphinxstyleliteralemphasis{int}) -- Height of meshgrid.

\item {} 
\sphinxstyleliteralstrong{grid\_size} (\sphinxstyleliteralemphasis{int}) -- Distance between grid lines.

\end{itemize}

\item[{Returns}] \leavevmode
\begin{itemize}
\item {} 
\sphinxstylestrong{grid\_points} (\sphinxstyleemphasis{ndarray}) -- Array of points of the grid

\item {} 
\sphinxstylestrong{quads} (\sphinxstyleemphasis{ndarray}) -- quads spanning the grid

\item {} 
\sphinxstylestrong{m} (\sphinxstyleemphasis{int}) -- dimension of the grid

\end{itemize}


\end{description}\end{quote}

\end{fulllineitems}

\index{construct\_mesh\_energy() (in module image\_aaap)}

\begin{fulllineitems}
\phantomsection\label{\detokenize{image_aaap:image_aaap.construct_mesh_energy}}\pysiglinewithargsret{\sphinxcode{image\_aaap.}\sphinxbfcode{construct\_mesh\_energy}}{\emph{grid\_points}, \emph{quads}, \emph{deform\_energy\_weights}}{}
Create quadratic energy matrix for aaap deformation of quad mesh.
\begin{quote}\begin{description}
\item[{Parameters}] \leavevmode\begin{itemize}
\item {} 
\sphinxstyleliteralstrong{grid\_points} (\sphinxstyleliteralemphasis{ndarray}) -- Array of points spanning the mesh.

\item {} 
\sphinxstyleliteralstrong{quads} (\sphinxstyleliteralemphasis{ndarray}) -- Index of grid points yielding quadrangulation of mesh.

\item {} 
\sphinxstyleliteralstrong{deform\_energy\_weights} (\sphinxstyleliteralemphasis{ndarray}) -- Weighting affinity of warping.
{[}alpha, beta, 0,0{]}. See paper for details.

\end{itemize}

\item[{Returns}] \leavevmode
\sphinxstylestrong{L} -- Sparse matrixs coresponding to aaap quadratic energy.

\item[{Return type}] \leavevmode
csc\_matrix

\end{description}\end{quote}

\end{fulllineitems}

\index{deform\_aaap() (in module image\_aaap)}

\begin{fulllineitems}
\phantomsection\label{\detokenize{image_aaap:image_aaap.deform_aaap}}\pysiglinewithargsret{\sphinxcode{image\_aaap.}\sphinxbfcode{deform\_aaap}}{\emph{x}, \emph{Asrc}, \emph{pdst}, \emph{L}, \emph{line\_constraint\_type}}{}
AAAP/ASAP deform a quadmesh with line constraints
\begin{quote}\begin{description}
\item[{Parameters}] \leavevmode\begin{itemize}
\item {} 
\sphinxstyleliteralstrong{x} (\sphinxstyleliteralemphasis{ndarray}) -- Geometry of the original quadmesh.

\item {} 
\sphinxstyleliteralstrong{Asrc} (\sphinxstyleliteralemphasis{csc\_matrix}) -- Matrix that express lines (sampled points on lines) as linear
combinations of x.

\item {} 
\sphinxstyleliteralstrong{pdst} (\sphinxstyleliteralemphasis{list}) -- Target positions of the lines (sampled points on them), each
cell element corresponds to one line.

\item {} 
\sphinxstyleliteralstrong{L} (\sphinxstyleliteralemphasis{csc\_matrix}) -- AAAP/ASAP energy of the quadmesh.

\item {} 
\sphinxstyleliteralstrong{line\_constraint\_type} (\sphinxstyleliteralemphasis{int}) -- Constraint type of each line.

\end{itemize}

\item[{Returns}] \leavevmode
\sphinxstylestrong{y} -- Geometry of the deformed quadmesh.

\item[{Return type}] \leavevmode
ndarray

\end{description}\end{quote}

\end{fulllineitems}

\index{sample\_lines() (in module image\_aaap)}

\begin{fulllineitems}
\phantomsection\label{\detokenize{image_aaap:image_aaap.sample_lines}}\pysiglinewithargsret{\sphinxcode{image\_aaap.}\sphinxbfcode{sample\_lines}}{\emph{src\_lines}, \emph{dst\_lines}, \emph{sample\_rate}}{}
Samples points from line pairs.
\begin{quote}\begin{description}
\item[{Parameters}] \leavevmode\begin{itemize}
\item {} 
\sphinxstyleliteralstrong{src\_lines} (\sphinxstyleliteralemphasis{list}) -- List of lines in source image.

\item {} 
\sphinxstyleliteralstrong{dst\_lines} (\sphinxstyleliteralemphasis{list}) -- List of lines in destination image.

\item {} 
\sphinxstyleliteralstrong{sample\_rate} (\sphinxstyleliteralemphasis{float}) -- Distance between sampled line points.

\end{itemize}

\item[{Returns}] \leavevmode
\begin{itemize}
\item {} 
\sphinxstylestrong{p1} (\sphinxstyleemphasis{ndarray}) -- List of points in src\_img.

\item {} 
\sphinxstylestrong{p2} (\sphinxstyleemphasis{list}) -- List of points in src\_img.

\end{itemize}


\end{description}\end{quote}

\end{fulllineitems}



\chapter{image\_aaap\_main module}
\label{\detokenize{image_aaap_main:image-aaap-main-module}}\label{\detokenize{image_aaap_main:module-image_aaap_main}}\label{\detokenize{image_aaap_main::doc}}\index{image\_aaap\_main (module)}
Wrapper for As-Affine-As-Possible Warping as described in
`Generalized As-Similar-As-Possible Warping with
Applications in Digital Photography' by Chen and Gotsman.
\index{aaap\_morph() (in module image\_aaap\_main)}

\begin{fulllineitems}
\phantomsection\label{\detokenize{image_aaap_main:image_aaap_main.aaap_morph}}\pysiglinewithargsret{\sphinxcode{image\_aaap\_main.}\sphinxbfcode{aaap\_morph}}{\emph{src\_img}, \emph{dst\_img}, \emph{src\_lines}, \emph{dst\_lines}, \emph{grid\_size=15}, \emph{line\_constraint\_type=2}, \emph{deform\_energy\_weights=array({[} 1.}, \emph{0.01}, \emph{0.}, \emph{0.  {]})}, \emph{n\_samples\_per\_grid=1}, \emph{scale\_factor=1}, \emph{show\_frame=False}, \emph{draw\_grid\_f=False}}{}
Warp src image as affine as possible under given line constraints.
\begin{quote}\begin{description}
\item[{Parameters}] \leavevmode\begin{itemize}
\item {} 
\sphinxstyleliteralstrong{src\_img} (\sphinxstyleliteralemphasis{ndarray}) -- Source image which will be warped to match destination image.

\item {} 
\sphinxstyleliteralstrong{dst\_img} (\sphinxstyleliteralemphasis{ndarray}) -- Destination image which is only scaled.

\item {} 
\sphinxstyleliteralstrong{src\_lines} (\sphinxstyleliteralemphasis{list}) -- User drawn lines in the source image.

\item {} 
\sphinxstyleliteralstrong{dst\_lines} (\sphinxstyleliteralemphasis{list}) -- User drawn lines in the destination image.

\item {} 
\sphinxstyleliteralstrong{grid\_size} (\sphinxstyleliteralemphasis{int}) -- Distance between grid lines in pixels. (Default value = 15)

\item {} 
\sphinxstyleliteralstrong{line\_constraint\_type} (\sphinxstyleliteralemphasis{int}) -- 0: Fixed discretisation of lines.
1: Flexible discretisations, points can move on line but endpoints
are fixed.
2: Flexible discretisation, all points including endpoints can move
on line. (Default value = 2)

\item {} 
\sphinxstyleliteralstrong{deform\_energy\_weights} (\sphinxstyleliteralemphasis{ndarray}) -- Weighting affinity of warping.
{[}alpha, beta, 0,0{]}. See paper for details.
(Default value = np.array({[}1,0.0100,0,0{]})

\item {} 
\sphinxstyleliteralstrong{deform\_energy\_weights} -- Weighting affinity of warping.
{[}alpha, beta, 0,0{]}. See paper for details.

\item {} 
\sphinxstyleliteralstrong{n\_samples\_per\_grid} (\sphinxstyleliteralemphasis{int}) -- Number of line discretization points per grid block.
(Default value = 1)

\item {} 
\sphinxstyleliteralstrong{scale\_factor} (\sphinxstyleliteralemphasis{int}) -- Scaling factor for first image and both line lists.
Second image is not scaled since not used for computation.
(Default value = 1)

\item {} 
\sphinxstyleliteralstrong{show\_frame} (\sphinxstyleliteralemphasis{bool}) -- True: Draw frame arround cropped area but do not crop.
False: Crop images.

\item {} 
\sphinxstyleliteralstrong{draw\_grid\_f} (\sphinxstyleliteralemphasis{bool}) -- True: Draw the aaap grid on the return images, False: Well...

\end{itemize}

\item[{Returns}] \leavevmode
\begin{itemize}
\item {} 
\sphinxstylestrong{src\_img\_morphed} (\sphinxstyleemphasis{ndarray}) -- Warped and cropped source image.

\item {} 
\sphinxstylestrong{dst\_img\_cropped} (\sphinxstyleemphasis{ndarray}) -- Destination image cropped to same size as src\_img.

\item {} 
\sphinxstylestrong{src\_img\_cropped} (\sphinxstyleemphasis{ndarray}) -- Source image cropped to evaluate warp.

\end{itemize}


\end{description}\end{quote}

\end{fulllineitems}



\chapter{image\_draw\_grid module}
\label{\detokenize{image_draw_grid:image-draw-grid-module}}\label{\detokenize{image_draw_grid::doc}}\label{\detokenize{image_draw_grid:module-image_draw_grid}}\index{image\_draw\_grid (module)}
Function to draw a grid on an image. Used to show the grid deformation of
aaap-morphing.
\index{draw\_grid() (in module image\_draw\_grid)}

\begin{fulllineitems}
\phantomsection\label{\detokenize{image_draw_grid:image_draw_grid.draw_grid}}\pysiglinewithargsret{\sphinxcode{image\_draw\_grid.}\sphinxbfcode{draw\_grid}}{\emph{img}, \emph{grid\_points}, \emph{quad\_indices}}{}
Draws line grid on given image.
\begin{quote}\begin{description}
\item[{Parameters}] \leavevmode\begin{itemize}
\item {} 
\sphinxstyleliteralstrong{img} (\sphinxstyleliteralemphasis{ndarray}) -- Image on which to draw.

\item {} 
\sphinxstyleliteralstrong{grid\_points} (\sphinxstyleliteralemphasis{ndarray}) -- List of cornerpoints of the grid.

\item {} 
\sphinxstyleliteralstrong{quad\_indices} (\sphinxstyleliteralemphasis{ndarray}) -- List of indices of quads.

\end{itemize}

\end{description}\end{quote}

\end{fulllineitems}



\chapter{image\_gabor module}
\label{\detokenize{image_gabor:image-gabor-module}}\label{\detokenize{image_gabor:module-image_gabor}}\label{\detokenize{image_gabor::doc}}\index{image\_gabor (module)}
Adapation of gabor\_threads.py from openCV/samples/python/
Filters an image with a set of gabor filters.
\index{build\_filters() (in module image\_gabor)}

\begin{fulllineitems}
\phantomsection\label{\detokenize{image_gabor:image_gabor.build_filters}}\pysiglinewithargsret{\sphinxcode{image\_gabor.}\sphinxbfcode{build\_filters}}{}{}
Builds collection of gabor filters.
\begin{quote}\begin{description}
\item[{Returns}] \leavevmode
\sphinxstylestrong{filters} -- List of gabor filters.

\item[{Return type}] \leavevmode
list

\end{description}\end{quote}

\end{fulllineitems}

\index{get\_gabor() (in module image\_gabor)}

\begin{fulllineitems}
\phantomsection\label{\detokenize{image_gabor:image_gabor.get_gabor}}\pysiglinewithargsret{\sphinxcode{image\_gabor.}\sphinxbfcode{get\_gabor}}{\emph{img}}{}
Returns gabor filtered image of a given image.
\begin{quote}\begin{description}
\item[{Parameters}] \leavevmode
\sphinxstyleliteralstrong{img} (\sphinxstyleliteralemphasis{ndarray}) -- Image to be filtered.

\item[{Returns}] \leavevmode
\sphinxstylestrong{img} -- The filtered image.

\item[{Return type}] \leavevmode
ndarray

\end{description}\end{quote}

\end{fulllineitems}

\index{process() (in module image\_gabor)}

\begin{fulllineitems}
\phantomsection\label{\detokenize{image_gabor:image_gabor.process}}\pysiglinewithargsret{\sphinxcode{image\_gabor.}\sphinxbfcode{process}}{\emph{img}, \emph{filters}}{}
Filters image by several gabor filters from a list.
\begin{quote}\begin{description}
\item[{Parameters}] \leavevmode\begin{itemize}
\item {} 
\sphinxstyleliteralstrong{img} (\sphinxstyleliteralemphasis{ndarray}) -- To be filtered image.

\item {} 
\sphinxstyleliteralstrong{filters} (\sphinxstyleliteralemphasis{list}) -- Gabor filters.

\end{itemize}

\item[{Returns}] \leavevmode
\sphinxstylestrong{accum} -- Gaborfiltered image.

\item[{Return type}] \leavevmode
ndarray

\end{description}\end{quote}

\end{fulllineitems}

\index{process\_threaded() (in module image\_gabor)}

\begin{fulllineitems}
\phantomsection\label{\detokenize{image_gabor:image_gabor.process_threaded}}\pysiglinewithargsret{\sphinxcode{image\_gabor.}\sphinxbfcode{process\_threaded}}{\emph{img}, \emph{filters}, \emph{threadn=4}}{}
Starts threated gabor filtering of image.
\begin{quote}\begin{description}
\item[{Parameters}] \leavevmode\begin{itemize}
\item {} 
\sphinxstyleliteralstrong{img} (\sphinxstyleliteralemphasis{ndarray}) -- To be filtered image.

\item {} 
\sphinxstyleliteralstrong{filters} (\sphinxstyleliteralemphasis{list}) -- Gabor filters.

\item {} 
\sphinxstyleliteralstrong{threadn} (\sphinxstyleliteralemphasis{int}) -- Number of threads for multiprocessing. (Default value = 4)

\end{itemize}

\item[{Returns}] \leavevmode
\sphinxstylestrong{accum} -- Gaborfiltered image.

\item[{Return type}] \leavevmode
ndarray

\end{description}\end{quote}

\end{fulllineitems}



\chapter{image\_helpers module}
\label{\detokenize{image_helpers:image-helpers-module}}\label{\detokenize{image_helpers::doc}}\label{\detokenize{image_helpers:module-image_helpers}}\index{image\_helpers (module)}
Collections of image processing function used throughout the project.
\index{adaptive\_thresh() (in module image\_helpers)}

\begin{fulllineitems}
\phantomsection\label{\detokenize{image_helpers:image_helpers.adaptive_thresh}}\pysiglinewithargsret{\sphinxcode{image\_helpers.}\sphinxbfcode{adaptive\_thresh}}{\emph{img}}{}
Thresholds given image adaptive.
\begin{quote}\begin{description}
\item[{Parameters}] \leavevmode
\sphinxstyleliteralstrong{img} (\sphinxstyleliteralemphasis{ndarray}) -- Image to be thresholded.

\item[{Returns}] \leavevmode
Thresholded grey image.

\item[{Return type}] \leavevmode
ndarray

\end{description}\end{quote}

\end{fulllineitems}

\index{do\_scale() (in module image\_helpers)}

\begin{fulllineitems}
\phantomsection\label{\detokenize{image_helpers:image_helpers.do_scale}}\pysiglinewithargsret{\sphinxcode{image\_helpers.}\sphinxbfcode{do\_scale}}{\emph{img1}, \emph{img2}, \emph{lines\_img1}, \emph{lines\_img2}, \emph{points\_img1}, \emph{points\_img2}, \emph{scale\_img1}, \emph{scale\_img2}, \emph{scale\_factor}}{}
Scales an image, line and point pair.
Uses a scale factor per image and one global factor.
\begin{quote}\begin{description}
\item[{Parameters}] \leavevmode\begin{itemize}
\item {} 
\sphinxstyleliteralstrong{img1} (\sphinxstyleliteralemphasis{ndarray}) -- First image to be scaled.

\item {} 
\sphinxstyleliteralstrong{img2} (\sphinxstyleliteralemphasis{ndarray}) -- Second image to be scaled.

\item {} 
\sphinxstyleliteralstrong{lines\_img1} (\sphinxstyleliteralemphasis{list}) -- First list of lines to be scaled.

\item {} 
\sphinxstyleliteralstrong{lines\_img2} (\sphinxstyleliteralemphasis{list}) -- Second list of lines to be scaled.

\item {} 
\sphinxstyleliteralstrong{points\_img1} (\sphinxstyleliteralemphasis{list}) -- First list of points to be scaled.

\item {} 
\sphinxstyleliteralstrong{points\_img2} (\sphinxstyleliteralemphasis{list}) -- Second list of points to be scaled.

\item {} 
\sphinxstyleliteralstrong{scale\_img1} (\sphinxstyleliteralemphasis{float}) -- Scale factor for first image/lines/points.

\item {} 
\sphinxstyleliteralstrong{scale\_img2} (\sphinxstyleliteralemphasis{float}) -- Scale factor for second image/lines/points.

\item {} 
\sphinxstyleliteralstrong{scale\_factor} (\sphinxstyleliteralemphasis{float}) -- global scale factor.

\end{itemize}

\item[{Returns}] \leavevmode
\begin{itemize}
\item {} 
\sphinxstylestrong{img1} (\sphinxstyleemphasis{ndarray}) -- Scaled first image.

\item {} 
\sphinxstylestrong{img2} (\sphinxstyleemphasis{ndarray}) -- Scaled second image.

\item {} 
\sphinxstylestrong{lines\_img1} (\sphinxstyleemphasis{list}) -- Scaled first list of lines.

\item {} 
\sphinxstylestrong{lines\_img2} (\sphinxstyleemphasis{list}) -- Scaled second list of lines.

\item {} 
\sphinxstylestrong{points\_img1} (\sphinxstyleemphasis{list}) -- Scaled first list of points.

\item {} 
\sphinxstylestrong{points\_img2} (\sphinxstyleemphasis{list}) -- Scaled second list of points.

\item {} 
\sphinxstylestrong{scale\_img1} (\sphinxstyleemphasis{float}) -- Scale factor of first image times global scale factor.

\item {} 
\sphinxstylestrong{scale\_img2} (\sphinxstyleemphasis{float}) -- Scale factor of second image times global scale factor.

\end{itemize}


\end{description}\end{quote}

\end{fulllineitems}

\index{draw\_circle() (in module image\_helpers)}

\begin{fulllineitems}
\phantomsection\label{\detokenize{image_helpers:image_helpers.draw_circle}}\pysiglinewithargsret{\sphinxcode{image\_helpers.}\sphinxbfcode{draw\_circle}}{\emph{img}, \emph{center}, \emph{color=(255}, \emph{255}, \emph{255)}}{}
Draws circle on given image.
\begin{quote}\begin{description}
\item[{Parameters}] \leavevmode\begin{itemize}
\item {} 
\sphinxstyleliteralstrong{img} (\sphinxstyleliteralemphasis{ndarray}) -- Image to be drawn on.

\item {} 
\sphinxstyleliteralstrong{center} (\sphinxstyleliteralemphasis{tuple}) -- Center of circle

\item {} 
\sphinxstyleliteralstrong{color} (\sphinxstyleliteralemphasis{tuple}) -- Color of the line. If no color given line is white.
(Default value = (255,255,255))

\end{itemize}

\end{description}\end{quote}

\end{fulllineitems}

\index{draw\_frame() (in module image\_helpers)}

\begin{fulllineitems}
\phantomsection\label{\detokenize{image_helpers:image_helpers.draw_frame}}\pysiglinewithargsret{\sphinxcode{image\_helpers.}\sphinxbfcode{draw\_frame}}{\emph{img}, \emph{x\_min}, \emph{x\_max}, \emph{y\_min}, \emph{y\_max}}{}
Draws a frame on a given image.
Used to display cropping lines
\begin{quote}\begin{description}
\item[{Parameters}] \leavevmode\begin{itemize}
\item {} 
\sphinxstyleliteralstrong{img} (\sphinxstyleliteralemphasis{ndarray}) -- Image on which frame is drawn

\item {} 
\sphinxstyleliteralstrong{x\_min} (\sphinxstyleliteralemphasis{int}) -- X coordinate of smaller point of rectangle

\item {} 
\sphinxstyleliteralstrong{y\_min} (\sphinxstyleliteralemphasis{int}) -- Y coordinate of smaller point of rectangle

\item {} 
\sphinxstyleliteralstrong{x\_max} (\sphinxstyleliteralemphasis{int}) -- X coordinate of bigger point of rectangle

\item {} 
\sphinxstyleliteralstrong{y\_max} (\sphinxstyleliteralemphasis{int}) -- Y coordinate of bigger point of rectangle

\end{itemize}

\end{description}\end{quote}

\end{fulllineitems}

\index{draw\_line() (in module image\_helpers)}

\begin{fulllineitems}
\phantomsection\label{\detokenize{image_helpers:image_helpers.draw_line}}\pysiglinewithargsret{\sphinxcode{image\_helpers.}\sphinxbfcode{draw\_line}}{\emph{img}, \emph{start}, \emph{end}, \emph{color=(255}, \emph{255}, \emph{255)}, \emph{l\_number=-1}}{}
Draws line and line number on given image.
\begin{quote}\begin{description}
\item[{Parameters}] \leavevmode\begin{itemize}
\item {} 
\sphinxstyleliteralstrong{img} (\sphinxstyleliteralemphasis{ndarray}) -- Image to be drawn on.

\item {} 
\sphinxstyleliteralstrong{start} (\sphinxstyleliteralemphasis{tuple}) -- Startpoint of line.

\item {} 
\sphinxstyleliteralstrong{end} (\sphinxstyleliteralemphasis{tuple}) -- Endpoint of line.

\item {} 
\sphinxstyleliteralstrong{color} (\sphinxstyleliteralemphasis{tuple}) -- Color of the line. If no color given line is white.
(Default value = (255,255,255))

\item {} 
\sphinxstyleliteralstrong{l\_number} (\sphinxstyleliteralemphasis{int}) -- Linenumber. If no number given only line is drawn.
(Default value = -1)

\end{itemize}

\end{description}\end{quote}

\end{fulllineitems}

\index{draw\_rectangle() (in module image\_helpers)}

\begin{fulllineitems}
\phantomsection\label{\detokenize{image_helpers:image_helpers.draw_rectangle}}\pysiglinewithargsret{\sphinxcode{image\_helpers.}\sphinxbfcode{draw\_rectangle}}{\emph{img}, \emph{start}, \emph{end}, \emph{color=(255}, \emph{255}, \emph{255)}}{}
Draws rectangle on given image.
\begin{quote}\begin{description}
\item[{Parameters}] \leavevmode\begin{itemize}
\item {} 
\sphinxstyleliteralstrong{img} (\sphinxstyleliteralemphasis{ndarray}) -- Image to be drawn on.

\item {} 
\sphinxstyleliteralstrong{start} (\sphinxstyleliteralemphasis{tuple}) -- Startpoint of rectangle.

\item {} 
\sphinxstyleliteralstrong{end} (\sphinxstyleliteralemphasis{tuple}) -- Endpoint of rectangle.

\item {} 
\sphinxstyleliteralstrong{color} (\sphinxstyleliteralemphasis{tuple}) -- Color of the line. If no color given line is white.
(Default value = (255,255,255)

\end{itemize}

\end{description}\end{quote}

\end{fulllineitems}

\index{get\_crop\_idx() (in module image\_helpers)}

\begin{fulllineitems}
\phantomsection\label{\detokenize{image_helpers:image_helpers.get_crop_idx}}\pysiglinewithargsret{\sphinxcode{image\_helpers.}\sphinxbfcode{get\_crop\_idx}}{\emph{crop\_img}, \emph{scale=400}}{}
Computes crop indices based on alpha channel.
Searches biggest white rectangle in alpha channel.
\begin{quote}\begin{description}
\item[{Parameters}] \leavevmode\begin{itemize}
\item {} 
\sphinxstyleliteralstrong{crop\_img} (\sphinxstyleliteralemphasis{ndarray}) -- to be cropped image with alpha channel

\item {} 
\sphinxstyleliteralstrong{scale} (\sphinxstyleliteralemphasis{int}) -- Downsample image by img size / scale to speed up (Default value = 400)

\end{itemize}

\item[{Returns}] \leavevmode
Cropindices {[}x\_min, y\_min, x\_max, y\_max{]}

\item[{Return type}] \leavevmode
list

\end{description}\end{quote}

\end{fulllineitems}

\index{lce() (in module image\_helpers)}

\begin{fulllineitems}
\phantomsection\label{\detokenize{image_helpers:image_helpers.lce}}\pysiglinewithargsret{\sphinxcode{image\_helpers.}\sphinxbfcode{lce}}{\emph{img}, \emph{kernel=11}, \emph{amount=0.5}}{}
Local Contrast Enhancement by unsharp mask.
From the value channel of the image in hsv color space a gaussian blured
version is subtracted.
\begin{quote}\begin{description}
\item[{Parameters}] \leavevmode\begin{itemize}
\item {} 
\sphinxstyleliteralstrong{img} (\sphinxstyleliteralemphasis{ndarray}) -- BGR-Image which is enhanced

\item {} 
\sphinxstyleliteralstrong{kernel} (\sphinxstyleliteralemphasis{int}) -- Size of the gaussian kernel. (Default value = 11)

\item {} 
\sphinxstyleliteralstrong{amount} (\sphinxstyleliteralemphasis{float}) -- Strength of the contrast enhancment. (Default value = 0.5)

\end{itemize}

\item[{Returns}] \leavevmode
\sphinxstylestrong{img\_bgr} -- Contrast enhanced np.float32 BGR-Image, values between 0, 255.

\item[{Return type}] \leavevmode
ndarray

\end{description}\end{quote}

\end{fulllineitems}

\index{line\_intersect() (in module image\_helpers)}

\begin{fulllineitems}
\phantomsection\label{\detokenize{image_helpers:image_helpers.line_intersect}}\pysiglinewithargsret{\sphinxcode{image\_helpers.}\sphinxbfcode{line\_intersect}}{\emph{a1}, \emph{a2}, \emph{b1}, \emph{b2}}{}
Compute intersection of two lines.
All input points have to be float.
Returns startpoint of second line if lines are parallel.
\begin{quote}\begin{description}
\item[{Parameters}] \leavevmode\begin{itemize}
\item {} 
\sphinxstyleliteralstrong{a1} (\sphinxstyleliteralemphasis{ndarray}) -- Startpoint first line.

\item {} 
\sphinxstyleliteralstrong{a2} (\sphinxstyleliteralemphasis{ndarray}) -- Endpoint first line.

\item {} 
\sphinxstyleliteralstrong{b1} (\sphinxstyleliteralemphasis{ndarray}) -- Startpoint second line.

\item {} 
\sphinxstyleliteralstrong{b2} (\sphinxstyleliteralemphasis{ndarray}) -- Endpoint second line.

\end{itemize}

\item[{Returns}] \leavevmode
intersection point.

\item[{Return type}] \leavevmode
ndarray

\end{description}\end{quote}

\end{fulllineitems}

\index{pint() (in module image\_helpers)}

\begin{fulllineitems}
\phantomsection\label{\detokenize{image_helpers:image_helpers.pint}}\pysiglinewithargsret{\sphinxcode{image\_helpers.}\sphinxbfcode{pint}}{\emph{p}}{}
\end{fulllineitems}

\index{scale() (in module image\_helpers)}

\begin{fulllineitems}
\phantomsection\label{\detokenize{image_helpers:image_helpers.scale}}\pysiglinewithargsret{\sphinxcode{image\_helpers.}\sphinxbfcode{scale}}{\emph{img1}, \emph{img2}, \emph{lines\_img1}, \emph{lines\_img2}, \emph{points\_img1}, \emph{points\_img2}, \emph{scale\_factor=1}}{}
Upscales the smaller image and coresponding lines/points of two given images.
If scale factor is given all entities are scaled by this factor.
Aspect ratio is preserved, blank space is filled with zeros.
\begin{quote}\begin{description}
\item[{Parameters}] \leavevmode\begin{itemize}
\item {} 
\sphinxstyleliteralstrong{img1} (\sphinxstyleliteralemphasis{ndarray}) -- Image 1.

\item {} 
\sphinxstyleliteralstrong{img2} (\sphinxstyleliteralemphasis{ndarray}) -- Image 2.

\item {} 
\sphinxstyleliteralstrong{lines\_img\_1} (\sphinxstyleliteralemphasis{list}) -- Lines in image 1.

\item {} 
\sphinxstyleliteralstrong{lines\_img\_2} (\sphinxstyleliteralemphasis{list}) -- Lines in image 2.

\item {} 
\sphinxstyleliteralstrong{points\_img\_1} (\sphinxstyleliteralemphasis{list}) -- Points in image 1.

\item {} 
\sphinxstyleliteralstrong{points\_img\_2} (\sphinxstyleliteralemphasis{list}) -- Points in image 2.

\item {} 
\sphinxstyleliteralstrong{scale\_factor} (\sphinxstyleliteralemphasis{int}) -- Scaling factor for image/line/point pair.
(Default value = 1)

\end{itemize}

\item[{Returns}] \leavevmode
\begin{itemize}
\item {} 
\sphinxstylestrong{img1} (\sphinxstyleemphasis{ndarray}) -- If img1 is bigger returns img1 else scaled img1.

\item {} 
\sphinxstylestrong{img2} (\sphinxstyleemphasis{ndarray}) -- If img2 is bigger returns img2 else scaled img2.

\item {} 
\sphinxstylestrong{lines\_img\_1} (\sphinxstyleemphasis{list}) -- Lines in image 1, scaled if img1 is scaled.

\item {} 
\sphinxstylestrong{lines\_img\_2} (\sphinxstyleemphasis{list}) -- Lines in image 2, scaled if img2 is scaled.

\item {} 
\sphinxstylestrong{points\_img\_1} (\sphinxstyleemphasis{list}) -- Points in image 1, scaled if img1 is scaled.

\item {} 
\sphinxstylestrong{points\_img\_2} (\sphinxstyleemphasis{list}) -- Points in image 2, scaled if img2 is scaled.

\item {} 
\sphinxstylestrong{scale\_factor\_img1} (\sphinxstyleemphasis{float}) -- Scale factor by which first image/lines/points were
scaled.

\item {} 
\sphinxstylestrong{scale\_factor\_img1} (\sphinxstyleemphasis{float}) -- Scale factor by which second image/lines/points were
scaled.

\item {} 
\sphinxstylestrong{x\_max} (\sphinxstyleemphasis{int}) -- After x\_max one image is padded with zeros in x direction.

\item {} 
\sphinxstylestrong{y\_max} (\sphinxstyleemphasis{int}) -- After y\_max one image is padded with zeros in y direction.

\end{itemize}


\end{description}\end{quote}

\end{fulllineitems}

\index{scale\_image\_lines\_points() (in module image\_helpers)}

\begin{fulllineitems}
\phantomsection\label{\detokenize{image_helpers:image_helpers.scale_image_lines_points}}\pysiglinewithargsret{\sphinxcode{image\_helpers.}\sphinxbfcode{scale\_image\_lines\_points}}{\emph{img}, \emph{lines}, \emph{points}, \emph{scale\_factor}}{}
Scales an image, points and lines in this image by a given scale factor.
\begin{quote}\begin{description}
\item[{Parameters}] \leavevmode\begin{itemize}
\item {} 
\sphinxstyleliteralstrong{img} (\sphinxstyleliteralemphasis{ndarray}) -- To be scaled image.

\item {} 
\sphinxstyleliteralstrong{lines} (\sphinxstyleliteralemphasis{list}) -- To be scaled lines.

\item {} 
\sphinxstyleliteralstrong{points} (\sphinxstyleliteralemphasis{list}) -- To be scaled points.

\item {} 
\sphinxstyleliteralstrong{scale\_factor} (\sphinxstyleliteralemphasis{int}) -- Factor by which image, lines and points are scaled.

\end{itemize}

\item[{Returns}] \leavevmode
\begin{itemize}
\item {} 
\sphinxstylestrong{img} (\sphinxstyleemphasis{ndarray}) -- Scaled image.

\item {} 
\sphinxstylestrong{lines} (\sphinxstyleemphasis{list}) -- Scaled lines.

\item {} 
\sphinxstylestrong{points} (\sphinxstyleemphasis{list}) -- Scaled points.

\end{itemize}


\end{description}\end{quote}

\end{fulllineitems}

\index{set\_verbose() (in module image\_helpers)}

\begin{fulllineitems}
\phantomsection\label{\detokenize{image_helpers:image_helpers.set_verbose}}\pysiglinewithargsret{\sphinxcode{image\_helpers.}\sphinxbfcode{set\_verbose}}{\emph{verbose}}{}
Sets the global verbosity function.
\begin{quote}\begin{description}
\item[{Parameters}] \leavevmode
\sphinxstyleliteralstrong{verbose} (\sphinxstyleliteralemphasis{bool}) -- If true verbose output, false no verbose output.

\end{description}\end{quote}

\end{fulllineitems}

\index{show\_image() (in module image\_helpers)}

\begin{fulllineitems}
\phantomsection\label{\detokenize{image_helpers:image_helpers.show_image}}\pysiglinewithargsret{\sphinxcode{image\_helpers.}\sphinxbfcode{show\_image}}{\emph{img}, \emph{name='img'}, \emph{x=1000}, \emph{y=1000}}{}
Resizes image and displays it in window with given name.
Constraints both sides of image by given length constraints.
\begin{quote}\begin{description}
\item[{Parameters}] \leavevmode\begin{itemize}
\item {} 
\sphinxstyleliteralstrong{img} (\sphinxstyleliteralemphasis{ndarray}) -- Image to be displayed.

\item {} 
\sphinxstyleliteralstrong{name} (\sphinxstyleliteralemphasis{string}) -- Name of openCV window in which image is displayed. (Default value = `img')

\item {} 
\sphinxstyleliteralstrong{x} (\sphinxstyleliteralemphasis{int}) -- Max size of image in x direction. (Default value = 1000)

\item {} 
\sphinxstyleliteralstrong{y} (\sphinxstyleliteralemphasis{int}) -- Max size of image in y direction. (Default value = 1000)

\end{itemize}

\item[{Returns}] \leavevmode
\begin{itemize}
\item {} 
\sphinxstylestrong{img\_d} (\sphinxstyleemphasis{ndarray}) -- Scaled image.

\item {} 
\sphinxstylestrong{scale} (\sphinxstyleemphasis{float}) -- Scale by which the image was scaled.

\end{itemize}


\end{description}\end{quote}

\end{fulllineitems}

\index{statistic\_canny() (in module image\_helpers)}

\begin{fulllineitems}
\phantomsection\label{\detokenize{image_helpers:image_helpers.statistic_canny}}\pysiglinewithargsret{\sphinxcode{image\_helpers.}\sphinxbfcode{statistic\_canny}}{\emph{img}, \emph{sigma=0.33}}{}
Edge detection on color image depending on image properties.
\begin{quote}\begin{description}
\item[{Parameters}] \leavevmode\begin{itemize}
\item {} 
\sphinxstyleliteralstrong{img} (\sphinxstyleliteralemphasis{ndarray}) -- Image on which to detect edges.

\item {} 
\sphinxstyleliteralstrong{sigma} (\sphinxstyleliteralemphasis{float}) -- Standard deviation. (Default value = 0.33)

\end{itemize}

\item[{Returns}] \leavevmode
Edge image.

\item[{Return type}] \leavevmode
ndarray

\end{description}\end{quote}

\end{fulllineitems}

\index{unsharp\_mask() (in module image\_helpers)}

\begin{fulllineitems}
\phantomsection\label{\detokenize{image_helpers:image_helpers.unsharp_mask}}\pysiglinewithargsret{\sphinxcode{image\_helpers.}\sphinxbfcode{unsharp\_mask}}{\emph{img}, \emph{sigma=1}, \emph{amount=0.8}}{}
Sharpends given image via unsharp mask.
\begin{quote}\begin{description}
\item[{Parameters}] \leavevmode\begin{itemize}
\item {} 
\sphinxstyleliteralstrong{img} (\sphinxstyleliteralemphasis{ndarray}) -- Image to be sharpened.

\item {} 
\sphinxstyleliteralstrong{sigma} (\sphinxstyleliteralemphasis{float}) -- Sigma of Gaussian kernel for bluring. (Default value = 1)

\item {} 
\sphinxstyleliteralstrong{amount} (\sphinxstyleliteralemphasis{float}) -- Amount of sharpening. (Default value = 0.8)

\end{itemize}

\item[{Returns}] \leavevmode
\sphinxstylestrong{img} -- Sharpened Image.

\item[{Return type}] \leavevmode
ndarray

\end{description}\end{quote}

\end{fulllineitems}

\index{vprint() (in module image\_helpers)}

\begin{fulllineitems}
\phantomsection\label{\detokenize{image_helpers:image_helpers.vprint}}\pysiglinewithargsret{\sphinxcode{image\_helpers.}\sphinxbfcode{vprint}}{\emph{*a}, \emph{**k}}{}
\end{fulllineitems}

\index{weighted\_average\_point() (in module image\_helpers)}

\begin{fulllineitems}
\phantomsection\label{\detokenize{image_helpers:image_helpers.weighted_average_point}}\pysiglinewithargsret{\sphinxcode{image\_helpers.}\sphinxbfcode{weighted\_average\_point}}{\emph{point1}, \emph{point2}, \emph{alpha}}{}
Return the average point between two points weighted by alpha.
\begin{quote}\begin{description}
\item[{Parameters}] \leavevmode\begin{itemize}
\item {} 
\sphinxstyleliteralstrong{point1} (\sphinxstyleliteralemphasis{tuple}) -- First point multiplied by 1 - alpha

\item {} 
\sphinxstyleliteralstrong{point2} (\sphinxstyleliteralemphasis{tuple}) -- Second point multiplied by alpha

\item {} 
\sphinxstyleliteralstrong{alpha} (\sphinxstyleliteralemphasis{float}) -- The weight.

\end{itemize}

\item[{Returns}] \leavevmode
Weighted point

\item[{Return type}] \leavevmode
tuple

\end{description}\end{quote}

\end{fulllineitems}



\chapter{image\_io module}
\label{\detokenize{image_io:module-image_io}}\label{\detokenize{image_io:image-io-module}}\label{\detokenize{image_io::doc}}\index{image\_io (module)}
Writes and saves point and line lists from and to json files
\index{read\_lines() (in module image\_io)}

\begin{fulllineitems}
\phantomsection\label{\detokenize{image_io:image_io.read_lines}}\pysiglinewithargsret{\sphinxcode{image\_io.}\sphinxbfcode{read\_lines}}{\emph{filename}}{}
Reads lines from files.
\begin{quote}\begin{description}
\item[{Parameters}] \leavevmode
\sphinxstyleliteralstrong{filename} (\sphinxstyleliteralemphasis{string}) -- Name of file.

\item[{Returns}] \leavevmode
\begin{itemize}
\item {} 
\sphinxstylestrong{src\_lines} (\sphinxstyleemphasis{list}) -- List of lines in src image. Empty list if unable to open file.

\item {} 
\sphinxstylestrong{dst\_lines} (\sphinxstyleemphasis{list}) -- List of lines in dst image. Empty list if unable to open file.

\end{itemize}


\end{description}\end{quote}

\end{fulllineitems}

\index{read\_points() (in module image\_io)}

\begin{fulllineitems}
\phantomsection\label{\detokenize{image_io:image_io.read_points}}\pysiglinewithargsret{\sphinxcode{image\_io.}\sphinxbfcode{read\_points}}{\emph{filename}}{}
Reads points from files.
\begin{quote}\begin{description}
\item[{Parameters}] \leavevmode
\sphinxstyleliteralstrong{filename} (\sphinxstyleliteralemphasis{string}) -- Name of file.

\item[{Returns}] \leavevmode
\begin{itemize}
\item {} 
\sphinxstylestrong{src\_points} (\sphinxstyleemphasis{list}) -- List of points in src image. Empty list if unable to open file.

\item {} 
\sphinxstylestrong{dst\_points} (\sphinxstyleemphasis{list}) -- List of points in dst image. Empty list if unable to open file.

\end{itemize}


\end{description}\end{quote}

\end{fulllineitems}

\index{write\_lines() (in module image\_io)}

\begin{fulllineitems}
\phantomsection\label{\detokenize{image_io:image_io.write_lines}}\pysiglinewithargsret{\sphinxcode{image\_io.}\sphinxbfcode{write\_lines}}{\emph{src\_lines}, \emph{dst\_lines}, \emph{filename}}{}
Writes linelists to file.
\begin{quote}\begin{description}
\item[{Parameters}] \leavevmode\begin{itemize}
\item {} 
\sphinxstyleliteralstrong{src\_lines} (\sphinxstyleliteralemphasis{list}) -- List of lines in src image.

\item {} 
\sphinxstyleliteralstrong{dst\_lines} (\sphinxstyleliteralemphasis{list}) -- List of lines in dst image.

\item {} 
\sphinxstyleliteralstrong{filename} (\sphinxstyleliteralemphasis{string}) -- Name of file.

\end{itemize}

\end{description}\end{quote}

\end{fulllineitems}

\index{write\_points() (in module image\_io)}

\begin{fulllineitems}
\phantomsection\label{\detokenize{image_io:image_io.write_points}}\pysiglinewithargsret{\sphinxcode{image\_io.}\sphinxbfcode{write\_points}}{\emph{src\_points}, \emph{dst\_points}, \emph{filename}}{}
Writes pointlists to file.
\begin{quote}\begin{description}
\item[{Parameters}] \leavevmode\begin{itemize}
\item {} 
\sphinxstyleliteralstrong{src\_points} (\sphinxstyleliteralemphasis{list}) -- List of points in src image.

\item {} 
\sphinxstyleliteralstrong{dst\_points} (\sphinxstyleliteralemphasis{list}) -- List of points in dst image.

\item {} 
\sphinxstyleliteralstrong{filename} (\sphinxstyleliteralemphasis{string}) -- Name of file.

\end{itemize}

\end{description}\end{quote}

\end{fulllineitems}



\chapter{image\_lines module}
\label{\detokenize{image_lines:image-lines-module}}\label{\detokenize{image_lines:module-image_lines}}\label{\detokenize{image_lines::doc}}\index{image\_lines (module)}
Functions to find interesting lines near user drawn lines as well as
corresponding lines.
\index{center\_of\_line() (in module image\_lines)}

\begin{fulllineitems}
\phantomsection\label{\detokenize{image_lines:image_lines.center_of_line}}\pysiglinewithargsret{\sphinxcode{image\_lines.}\sphinxbfcode{center\_of\_line}}{\emph{p1}, \emph{p2}}{}
\end{fulllineitems}

\index{get\_corresponding\_line() (in module image\_lines)}

\begin{fulllineitems}
\phantomsection\label{\detokenize{image_lines:image_lines.get_corresponding_line}}\pysiglinewithargsret{\sphinxcode{image\_lines.}\sphinxbfcode{get\_corresponding\_line}}{\emph{img1}, \emph{img2}, \emph{line1}, \emph{psd=15}, \emph{max\_lines\_to\_check=60}, \emph{template\_size=200}}{}
Return a corresponding line in a second image given a line in a first image.
Find max\_lines\_to\_check lines in patch arround line1 in img2. Compare
correspondence of found lines by template matching. Transform template in
img2 such that found line in img2 and given line in img1 allign. Compute
sum of squared differences and weight templates/lines.
\begin{quote}\begin{description}
\item[{Parameters}] \leavevmode\begin{itemize}
\item {} 
\sphinxstyleliteralstrong{img1} (\sphinxstyleliteralemphasis{ndarray}) -- Destination image in which line is already found

\item {} 
\sphinxstyleliteralstrong{img2} (\sphinxstyleliteralemphasis{ndarray}) -- Source image in which the corresponding line is searched

\item {} 
\sphinxstyleliteralstrong{line1} (\sphinxstyleliteralemphasis{list}) -- Line in img1 for which a corresponding line needs to be found

\item {} 
\sphinxstyleliteralstrong{psd} (\sphinxstyleliteralemphasis{int}) -- Patchsizedivisor (Default value = 15)

\item {} 
\sphinxstyleliteralstrong{max\_lines\_to\_check} (\sphinxstyleliteralemphasis{int}) -- Number of lines which correspondence to given line is evaluated.
(Default value = 60)

\item {} 
\sphinxstyleliteralstrong{template\_size} (\sphinxstyleliteralemphasis{int}) -- Size of template with which correspondence of line is determined.
(Default value = 200)

\end{itemize}

\item[{Returns}] \leavevmode
The correspoding line.

\item[{Return type}] \leavevmode
list

\end{description}\end{quote}

\end{fulllineitems}

\index{get\_line() (in module image\_lines)}

\begin{fulllineitems}
\phantomsection\label{\detokenize{image_lines:image_lines.get_line}}\pysiglinewithargsret{\sphinxcode{image\_lines.}\sphinxbfcode{get\_line}}{\emph{p1}, \emph{p2}, \emph{img}, \emph{psd=70}}{}
Search in image for nearest most similar line of a given line.
Search in H, S, and V for line in area around given line. Select line
with similar rotation and many supporting edge pixels.
\begin{quote}\begin{description}
\item[{Parameters}] \leavevmode\begin{itemize}
\item {} 
\sphinxstyleliteralstrong{p1} (\sphinxstyleliteralemphasis{ndarray}) -- Start point of user drawn line.

\item {} 
\sphinxstyleliteralstrong{p2} (\sphinxstyleliteralemphasis{ndarray}) -- End point of user drawn line.

\item {} 
\sphinxstyleliteralstrong{img} (\sphinxstyleliteralemphasis{ndarray}) -- Image in which nearest line in searched.

\item {} 
\sphinxstyleliteralstrong{psd} (\sphinxstyleliteralemphasis{iny}) -- Patch size divisor: Area around line in which similar line is
searched. Higher value = Smaller area (Default value = 70)

\end{itemize}

\item[{Returns}] \leavevmode
\sphinxstylestrong{{[}p1,p2{]}} -- List of two points of computed most similar line

\item[{Return type}] \leavevmode
list

\end{description}\end{quote}

\end{fulllineitems}

\index{get\_patch() (in module image\_lines)}

\begin{fulllineitems}
\phantomsection\label{\detokenize{image_lines:image_lines.get_patch}}\pysiglinewithargsret{\sphinxcode{image\_lines.}\sphinxbfcode{get\_patch}}{\emph{img}, \emph{p1}, \emph{p2}, \emph{psd=70}}{}
Return image patch arround a given line.
Return horizontal patch, a rectangular mask with same slope as the line and
offset between min point of patch and image
\begin{quote}\begin{description}
\item[{Parameters}] \leavevmode\begin{itemize}
\item {} 
\sphinxstyleliteralstrong{img} (\sphinxstyleliteralemphasis{ndarray}) -- Image from which the patch is generated

\item {} 
\sphinxstyleliteralstrong{p1} (\sphinxstyleliteralemphasis{tuple}) -- First point of the given line

\item {} 
\sphinxstyleliteralstrong{p2} (\sphinxstyleliteralemphasis{tuple}) -- Second point of the given line

\item {} 
\sphinxstyleliteralstrong{psd} (\sphinxstyleliteralemphasis{int}) -- Patchsizedivisor (Default value = 70) determining the patch size i.e.
the size arround the given line.

\end{itemize}

\item[{Returns}] \leavevmode
\begin{itemize}
\item {} 
\sphinxstylestrong{patch} (\sphinxstyleemphasis{ndarray}) -- Horizontal patch

\item {} 
\sphinxstylestrong{mask} (\sphinxstyleemphasis{ndarray}) -- Zero matrix of size patch with one rectangle determining the
actual patch arround the line.

\item {} 
\sphinxstylestrong{offset} (\sphinxstyleemphasis{ndarray}) -- Offset between min point of patch and min point of image.

\end{itemize}


\end{description}\end{quote}

\end{fulllineitems}

\index{get\_theta() (in module image\_lines)}

\begin{fulllineitems}
\phantomsection\label{\detokenize{image_lines:image_lines.get_theta}}\pysiglinewithargsret{\sphinxcode{image\_lines.}\sphinxbfcode{get\_theta}}{\emph{p1}, \emph{p2}}{}
Computes gradient angle of line.
\begin{quote}\begin{description}
\item[{Parameters}] \leavevmode\begin{itemize}
\item {} 
\sphinxstyleliteralstrong{p1} (\sphinxstyleliteralemphasis{ndarray}) -- First point of line.

\item {} 
\sphinxstyleliteralstrong{p2} (\sphinxstyleliteralemphasis{ndarray}) -- Second point of line.

\end{itemize}

\item[{Returns}] \leavevmode
\sphinxstylestrong{theta} -- Gradient angle.

\item[{Return type}] \leavevmode
float64

\end{description}\end{quote}

\end{fulllineitems}

\index{get\_transformed\_patch() (in module image\_lines)}

\begin{fulllineitems}
\phantomsection\label{\detokenize{image_lines:image_lines.get_transformed_patch}}\pysiglinewithargsret{\sphinxcode{image\_lines.}\sphinxbfcode{get\_transformed\_patch}}{\emph{patch}, \emph{p11}, \emph{p12}, \emph{p21}, \emph{p22}}{}
Rotates and translates given patch such that second line is mapped to a
first horizontal line.
\begin{quote}\begin{description}
\item[{Parameters}] \leavevmode\begin{itemize}
\item {} 
\sphinxstyleliteralstrong{patch} (\sphinxstyleliteralemphasis{ndarray}) -- Patch to be translated and rotated.

\item {} 
\sphinxstyleliteralstrong{p11} (\sphinxstyleliteralemphasis{tuple}) -- First point of first line.

\item {} 
\sphinxstyleliteralstrong{p12} (\sphinxstyleliteralemphasis{tuple}) -- Second point of first line.

\item {} 
\sphinxstyleliteralstrong{p21} (\sphinxstyleliteralemphasis{tuple}) -- First point of second line.

\item {} 
\sphinxstyleliteralstrong{p22} (\sphinxstyleliteralemphasis{tuple}) -- Second point of second line.

\end{itemize}

\item[{Returns}] \leavevmode
\sphinxstylestrong{patch} -- Transformed patch.

\item[{Return type}] \leavevmode
ndarray

\end{description}\end{quote}

\end{fulllineitems}

\index{get\_weighted\_lines() (in module image\_lines)}

\begin{fulllineitems}
\phantomsection\label{\detokenize{image_lines:image_lines.get_weighted_lines}}\pysiglinewithargsret{\sphinxcode{image\_lines.}\sphinxbfcode{get\_weighted\_lines}}{\emph{img}, \emph{mask}, \emph{p1\_o}, \emph{p2\_o}, \emph{range\_theta=0.2617993877991494}}{}
Returns sortet list of lines in proximity of a given line.
Lines are sorted by line quality.
\begin{quote}\begin{description}
\item[{Parameters}] \leavevmode\begin{itemize}
\item {} 
\sphinxstyleliteralstrong{img} (\sphinxstyleliteralemphasis{ndarray}) -- Image in which lines are searched.

\item {} 
\sphinxstyleliteralstrong{mask} (\sphinxstyleliteralemphasis{ndarray}) -- Mask applied to image to limit search area.

\item {} 
\sphinxstyleliteralstrong{p1\_o} (\sphinxstyleliteralemphasis{ndarray}) -- First point of original line.

\item {} 
\sphinxstyleliteralstrong{p2\_o} (\sphinxstyleliteralemphasis{ndarray}) -- Second point of original line.

\item {} 
\sphinxstyleliteralstrong{range\_theta} (\sphinxstyleliteralemphasis{float}) -- Limits how far new lines are rotated from original line.
(Default value = 1/12.*np.pi)

\end{itemize}

\item[{Returns}] \leavevmode
\sphinxstylestrong{best\_lines} -- Array of lines in descending order.

\item[{Return type}] \leavevmode
ndarray

\end{description}\end{quote}

\end{fulllineitems}

\index{lim\_line\_length() (in module image\_lines)}

\begin{fulllineitems}
\phantomsection\label{\detokenize{image_lines:image_lines.lim_line_length}}\pysiglinewithargsret{\sphinxcode{image\_lines.}\sphinxbfcode{lim\_line\_length}}{\emph{p1\_h}, \emph{p2\_h}, \emph{p1\_o}, \emph{p2\_o}}{}
Limits length of line h to linesegment o.
Computes a line segment of line h found by hough transform to the
length of user drawn line segment o by computing the normals at start
and end point of user drawn line segment and their intersection with
hough line.
\begin{quote}\begin{description}
\item[{Parameters}] \leavevmode\begin{itemize}
\item {} 
\sphinxstyleliteralstrong{p1\_h} (\sphinxstyleliteralemphasis{ndarray}) -- First point on hough line.

\item {} 
\sphinxstyleliteralstrong{p2\_h} (\sphinxstyleliteralemphasis{ndarray}) -- Second point on hough line.

\item {} 
\sphinxstyleliteralstrong{p1\_o} (\sphinxstyleliteralemphasis{ndarray}) -- Startpoint of user drawn line segment.

\item {} 
\sphinxstyleliteralstrong{p2\_o} (\sphinxstyleliteralemphasis{ndarray}) -- Endpoint of user drawn line segment.

\end{itemize}

\item[{Returns}] \leavevmode
\begin{itemize}
\item {} 
\sphinxstylestrong{z1} (\sphinxstyleemphasis{ndarray}) -- Startpoint of line segment of hough line.

\item {} 
\sphinxstylestrong{z2} (\sphinxstyleemphasis{ndarray}) -- Endpoint of line segment of hough line.

\end{itemize}


\end{description}\end{quote}

\end{fulllineitems}

\index{line\_detect() (in module image\_lines)}

\begin{fulllineitems}
\phantomsection\label{\detokenize{image_lines:image_lines.line_detect}}\pysiglinewithargsret{\sphinxcode{image\_lines.}\sphinxbfcode{line\_detect}}{\emph{img}, \emph{mask=None}, \emph{sigma=0.33}, \emph{magic\_n=10}, \emph{min\_theta=0}, \emph{max\_theta=3.141592653589793}}{}
Line detection on given image via Canny and Hough.
Lines are limited to have an angle between min\_theta and max\_theta.
\begin{quote}\begin{description}
\item[{Parameters}] \leavevmode\begin{itemize}
\item {} 
\sphinxstyleliteralstrong{img} (\sphinxstyleliteralemphasis{ndarray}) -- Image in which to detect lines.

\item {} 
\sphinxstyleliteralstrong{mask} (\sphinxstyleliteralemphasis{ndarray}) -- Matrix of image size. Detection of lines only where mask is
nonzero. (Default value = None)

\item {} 
\sphinxstyleliteralstrong{sigma} (\sphinxstyleliteralemphasis{float}) -- Sigma for adpative Canny edge detection. (Default value = 0.33)

\item {} 
\sphinxstyleliteralstrong{magic\_n} (\sphinxstyleliteralemphasis{int}) -- Offset for adaptive Canny edge detection. (Default value = 10)

\item {} 
\sphinxstyleliteralstrong{min\_theta} (\sphinxstyleliteralemphasis{float64}) -- Minimum line angle for hough line detection.
(Default value = 0)

\item {} 
\sphinxstyleliteralstrong{max\_theta} (\sphinxstyleliteralemphasis{float64}) -- Maximum line angle for hough line detection.
(Default value = np.pi)

\end{itemize}

\item[{Returns}] \leavevmode
\sphinxstylestrong{lines} -- Detected lines in Hesse normal form.

\item[{Return type}] \leavevmode
ndarray

\end{description}\end{quote}

\end{fulllineitems}

\index{weight\_lines() (in module image\_lines)}

\begin{fulllineitems}
\phantomsection\label{\detokenize{image_lines:image_lines.weight_lines}}\pysiglinewithargsret{\sphinxcode{image\_lines.}\sphinxbfcode{weight\_lines}}{\emph{patch\_p}, \emph{lines}, \emph{p1\_o}, \emph{p2\_o}, \emph{max\_delta=0.05}, \emph{number\_of\_lines=10}}{}
Weights hough lines by similarity to edges.
Generates line segments with length according to user drawn lines and
compares segments with edges in edge image. Select best matching line.
Only lines with similar orientation to user drawn line are regarded.
Only best ten lines fullfilling above criterium are considered.
\begin{quote}\begin{description}
\item[{Parameters}] \leavevmode\begin{itemize}
\item {} 
\sphinxstyleliteralstrong{patch\_p} (\sphinxstyleliteralemphasis{ndarray}) -- Edgeimage.

\item {} 
\sphinxstyleliteralstrong{lines} (\sphinxstyleliteralemphasis{ndarray}) -- List of lines.

\item {} 
\sphinxstyleliteralstrong{p1\_o} (\sphinxstyleliteralemphasis{ndarray}) -- Startpoint of user drawn line.

\item {} 
\sphinxstyleliteralstrong{p2\_o} (\sphinxstyleliteralemphasis{ndarray}) -- Endpoint of user drawn line.

\item {} 
\sphinxstyleliteralstrong{max\_delta} (\sphinxstyleliteralemphasis{float}) -- max angle between given line and lines to be weighted. (Default value = 0.05)

\item {} 
\sphinxstyleliteralstrong{number\_of\_lines} (\sphinxstyleliteralemphasis{int}) -- number of lines generated by first step (Default value = 10)

\end{itemize}

\item[{Returns}] \leavevmode
\begin{itemize}
\item {} 
\sphinxstylestrong{line\_segs} (\sphinxstyleemphasis{ndarray}) -- best lines.

\item {} 
\sphinxstylestrong{weights} (\sphinxstyleemphasis{ndarray}) -- weights of the best line segments.

\end{itemize}


\end{description}\end{quote}

\end{fulllineitems}



\chapter{image\_morphing module}
\label{\detokenize{image_morphing:module-image_morphing}}\label{\detokenize{image_morphing:image-morphing-module}}\label{\detokenize{image_morphing::doc}}\index{image\_morphing (module)}
Functions to moprh an image with src and dst quad mesh. Corners of the two
quad meshes are mapped to each other, area inbetween is interpolated witch
perspective transform.
\index{morph() (in module image\_morphing)}

\begin{fulllineitems}
\phantomsection\label{\detokenize{image_morphing:image_morphing.morph}}\pysiglinewithargsret{\sphinxcode{image\_morphing.}\sphinxbfcode{morph}}{\emph{src\_img}, \emph{points\_old}, \emph{points\_new}, \emph{quads}, \emph{grid\_size}, \emph{processes=1}}{}
Returns morphed image given points of old and new grid and quadindices.
Sourceimage is divided in stripes which are morphed parallel.
\begin{quote}\begin{description}
\item[{Parameters}] \leavevmode\begin{itemize}
\item {} 
\sphinxstyleliteralstrong{src\_img} (\sphinxstyleliteralemphasis{ndarray}) -- The image which will be morphed.

\item {} 
\sphinxstyleliteralstrong{points\_old} (\sphinxstyleliteralemphasis{ndarray}) -- Positions of grid points in src\_img.

\item {} 
\sphinxstyleliteralstrong{points\_new} (\sphinxstyleliteralemphasis{ndarray}) -- Positions to where the old points are moved.

\item {} 
\sphinxstyleliteralstrong{quads} (\sphinxstyleliteralemphasis{ndarray}) -- List of quad indices.

\item {} 
\sphinxstyleliteralstrong{grid\_size} (\sphinxstyleliteralemphasis{int}) -- Distance between grid lines.

\item {} 
\sphinxstyleliteralstrong{processes} (\sphinxstyleliteralemphasis{int}) -- Number of multiprocessing.Processes which are spawend. (Default value = 1)

\end{itemize}

\item[{Returns}] \leavevmode
\sphinxstylestrong{img\_morh} -- The morphed src\_img.

\item[{Return type}] \leavevmode
ndarray

\end{description}\end{quote}

\end{fulllineitems}

\index{morph\_process() (in module image\_morphing)}

\begin{fulllineitems}
\phantomsection\label{\detokenize{image_morphing:image_morphing.morph_process}}\pysiglinewithargsret{\sphinxcode{image\_morphing.}\sphinxbfcode{morph\_process}}{\emph{src\_img}, \emph{s\_x\_min}, \emph{shared\_dst}, \emph{dst\_shape}, \emph{points\_new}, \emph{points\_old}, \emph{quads}}{}
Multiprocessing process to morph image stripes.
\begin{quote}\begin{description}
\item[{Parameters}] \leavevmode\begin{itemize}
\item {} 
\sphinxstyleliteralstrong{src\_img} (\sphinxstyleliteralemphasis{ndarray}) -- Stripe of the original image which is to be morphed.

\item {} 
\sphinxstyleliteralstrong{s\_x\_min} (\sphinxstyleliteralemphasis{int64}) -- Offset of image stripe. Used to determine where to add stripe
in shared\_dst.

\item {} 
\sphinxstyleliteralstrong{shared\_dst} (\sphinxstyleliteralemphasis{SynchronizedArray}) -- Shared destination image in which morphed image is saved.

\item {} 
\sphinxstyleliteralstrong{dst\_shape} (\sphinxstyleliteralemphasis{tuple}) -- Size of the destination image.

\item {} 
\sphinxstyleliteralstrong{points\_new} (\sphinxstyleliteralemphasis{ndarray}) -- Corner locations of the quad grid in the destination image.

\item {} 
\sphinxstyleliteralstrong{points\_old} (\sphinxstyleliteralemphasis{ndarray}) -- Corner locations of the quad grid in the source image.

\item {} 
\sphinxstyleliteralstrong{quads} (\sphinxstyleliteralemphasis{ndarray}) -- Indices of corners of the quads constituiting the mesh.

\end{itemize}

\end{description}\end{quote}

\end{fulllineitems}

\index{to\_numpy\_array() (in module image\_morphing)}

\begin{fulllineitems}
\phantomsection\label{\detokenize{image_morphing:image_morphing.to_numpy_array}}\pysiglinewithargsret{\sphinxcode{image\_morphing.}\sphinxbfcode{to\_numpy\_array}}{\emph{mp\_arr}}{}
Create numpy array from multiprocessing array.
\begin{quote}\begin{description}
\item[{Parameters}] \leavevmode
\sphinxstyleliteralstrong{mp\_arr} (\sphinxstyleliteralemphasis{SynchronizedArray}) -- Multiprocessing input array.

\item[{Returns}] \leavevmode
Numpy output array.

\item[{Return type}] \leavevmode
ndarray

\end{description}\end{quote}

\end{fulllineitems}



\chapter{image\_perspective\_alignment module}
\label{\detokenize{image_perspective_alignment:image-perspective-alignment-module}}\label{\detokenize{image_perspective_alignment::doc}}\label{\detokenize{image_perspective_alignment:module-image_perspective_alignment}}\index{image\_perspective\_alignment (module)}
Functions to transform images, list of lines and list of points to match
another image with a perspective transform matrix. The later is computed from
two lists of points, which should be matched.
\index{perspective\_align() (in module image\_perspective\_alignment)}

\begin{fulllineitems}
\phantomsection\label{\detokenize{image_perspective_alignment:image_perspective_alignment.perspective_align}}\pysiglinewithargsret{\sphinxcode{image\_perspective\_alignment.}\sphinxbfcode{perspective\_align}}{\emph{img\_1}, \emph{img\_2}, \emph{points\_img\_1}, \emph{points\_img\_2}, \emph{lines\_img\_1}, \emph{lines\_img\_2}, \emph{alpha=None}}{}
Aligns the two images with the best matching perspective transform given
the two point lists. Points and lines in their list are transformed as well.
\begin{quote}\begin{description}
\item[{Parameters}] \leavevmode\begin{itemize}
\item {} 
\sphinxstyleliteralstrong{img\_1} (\sphinxstyleliteralemphasis{ndarray}) -- Image 1

\item {} 
\sphinxstyleliteralstrong{img\_2} (\sphinxstyleliteralemphasis{ndarray}) -- Image 2

\item {} 
\sphinxstyleliteralstrong{points\_img\_1} (\sphinxstyleliteralemphasis{list}) -- marked points in image 1

\item {} 
\sphinxstyleliteralstrong{points\_img\_2} (\sphinxstyleliteralemphasis{list}) -- coresponding points in image 2

\item {} 
\sphinxstyleliteralstrong{lines\_img\_1} (\sphinxstyleliteralemphasis{list}) -- marked lines in image 1

\item {} 
\sphinxstyleliteralstrong{lines\_img\_2} (\sphinxstyleliteralemphasis{list}) -- coresponding lins in image 2

\item {} 
\sphinxstyleliteralstrong{alpha} (\sphinxstyleliteralemphasis{float}) -- 0 = align img\_2 to img\_1, 1 = align img\_1 to img\_2, 0.5 align
img\_1 and img\_2 to mean and points and lines acordingly.
If alpha is None the smaller image is transformed to bigger one.
(Default value = None)

\end{itemize}

\item[{Returns}] \leavevmode
\begin{itemize}
\item {} 
\sphinxstylestrong{img\_1} (\sphinxstyleemphasis{ndarray}) -- perspective transformed img\_1

\item {} 
\sphinxstylestrong{img\_2} (\sphinxstyleemphasis{ndarray}) -- perspective transformed img\_2

\item {} 
\sphinxstylestrong{points\_img\_1} (\sphinxstyleemphasis{list}) -- perspective transformed points\_img\_1

\item {} 
\sphinxstylestrong{points\_img\_2} (\sphinxstyleemphasis{list}) -- perspective transformed points\_img\_2

\item {} 
\sphinxstylestrong{lines\_img\_1} (\sphinxstyleemphasis{list}) -- perspective transformed lines\_img\_1

\item {} 
\sphinxstylestrong{lines\_img\_2} (\sphinxstyleemphasis{list}) -- perspective transformed lines\_img\_2

\end{itemize}


\end{description}\end{quote}

\end{fulllineitems}

\index{transform\_lines() (in module image\_perspective\_alignment)}

\begin{fulllineitems}
\phantomsection\label{\detokenize{image_perspective_alignment:image_perspective_alignment.transform_lines}}\pysiglinewithargsret{\sphinxcode{image\_perspective\_alignment.}\sphinxbfcode{transform\_lines}}{\emph{lines}, \emph{transform\_matrix}}{}
Transforms a list of lines given a transformation matrix.
\begin{quote}\begin{description}
\item[{Parameters}] \leavevmode\begin{itemize}
\item {} 
\sphinxstyleliteralstrong{lines} (\sphinxstyleliteralemphasis{list}) -- The list of lines given as lists.

\item {} 
\sphinxstyleliteralstrong{transform\_matrix} (\sphinxstyleliteralemphasis{ndarray}) -- The transformation matrix.

\end{itemize}

\item[{Returns}] \leavevmode
\sphinxstylestrong{lines} -- Transformed lines in float32

\item[{Return type}] \leavevmode
list

\end{description}\end{quote}

\end{fulllineitems}

\index{transform\_points() (in module image\_perspective\_alignment)}

\begin{fulllineitems}
\phantomsection\label{\detokenize{image_perspective_alignment:image_perspective_alignment.transform_points}}\pysiglinewithargsret{\sphinxcode{image\_perspective\_alignment.}\sphinxbfcode{transform\_points}}{\emph{points}, \emph{transform\_matrix}}{}
Transforms a list of points given a transformation matrix.
\begin{quote}\begin{description}
\item[{Parameters}] \leavevmode\begin{itemize}
\item {} 
\sphinxstyleliteralstrong{points} (\sphinxstyleliteralemphasis{list}) -- the list of points given as touples

\item {} 
\sphinxstyleliteralstrong{transform\_matrix} (\sphinxstyleliteralemphasis{ndarray}) -- the transformation matrix

\end{itemize}

\item[{Returns}] \leavevmode
\sphinxstylestrong{point\_array} -- transformed points in int32

\item[{Return type}] \leavevmode
list

\end{description}\end{quote}

\end{fulllineitems}



\chapter{image\_sac module}
\label{\detokenize{image_sac:image-sac-module}}\label{\detokenize{image_sac::doc}}\label{\detokenize{image_sac:module-image_sac}}\index{image\_sac (module)}
Functions to find interesting points near user drawn points as well as
corresponding points.
\index{getCorespondingPoint() (in module image\_sac)}

\begin{fulllineitems}
\phantomsection\label{\detokenize{image_sac:image_sac.getCorespondingPoint}}\pysiglinewithargsret{\sphinxcode{image\_sac.}\sphinxbfcode{getCorespondingPoint}}{\emph{img1}, \emph{img2}, \emph{point}, \emph{template\_size\_s=101}}{}
Search for coresponding point on second image given a point in first image.

First possible matching corners are searched, then template matching at the
possible corner locations is done.
\begin{quote}\begin{description}
\item[{Parameters}] \leavevmode\begin{itemize}
\item {} 
\sphinxstyleliteralstrong{img1} (\sphinxstyleliteralemphasis{ndarray}) -- Image in which point was found.

\item {} 
\sphinxstyleliteralstrong{img2} (\sphinxstyleliteralemphasis{ndarray}) -- Image in which corresponding point is searched.

\item {} 
\sphinxstyleliteralstrong{point} (\sphinxstyleliteralemphasis{ndarray}) -- Location of found point in img1.

\item {} 
\sphinxstyleliteralstrong{template\_size\_1} (\sphinxstyleliteralemphasis{int}) -- Size of scaled template for template matching.

\end{itemize}

\item[{Returns}] \leavevmode
\sphinxstylestrong{point} -- The corresponding point.

\item[{Return type}] \leavevmode
ndarray

\end{description}\end{quote}

\end{fulllineitems}

\index{getPFromRectangleACorespondingP() (in module image\_sac)}

\begin{fulllineitems}
\phantomsection\label{\detokenize{image_sac:image_sac.getPFromRectangleACorespondingP}}\pysiglinewithargsret{\sphinxcode{image\_sac.}\sphinxbfcode{getPFromRectangleACorespondingP}}{\emph{img1}, \emph{img2}, \emph{point1}, \emph{point2}}{}
Wrapper for getPointFromRectangle and getCorespondingPoint.
\begin{quote}\begin{description}
\item[{Parameters}] \leavevmode\begin{itemize}
\item {} 
\sphinxstyleliteralstrong{img1} (\sphinxstyleliteralemphasis{ndarray}) -- Image in which point is searched.

\item {} 
\sphinxstyleliteralstrong{img2} (\sphinxstyleliteralemphasis{ndarray}) -- Image in which corresponding point is searched.

\item {} 
\sphinxstyleliteralstrong{point1} (\sphinxstyleliteralemphasis{tuple}) -- First corner of rectangle in which point is searched.

\item {} 
\sphinxstyleliteralstrong{point2} (\sphinxstyleliteralemphasis{tuple}) -- Second corner of rectangle in which point is searched.

\end{itemize}

\item[{Returns}] \leavevmode
\begin{itemize}
\item {} 
\sphinxstylestrong{returnPoint1} (\sphinxstyleemphasis{tuple}) -- Found point.

\item {} 
\sphinxstylestrong{returnPoint2} (\sphinxstyleemphasis{tuple}) -- Corresponding point.

\end{itemize}


\end{description}\end{quote}

\end{fulllineitems}

\index{getPointFromPoint() (in module image\_sac)}

\begin{fulllineitems}
\phantomsection\label{\detokenize{image_sac:image_sac.getPointFromPoint}}\pysiglinewithargsret{\sphinxcode{image\_sac.}\sphinxbfcode{getPointFromPoint}}{\emph{img}, \emph{point}}{}
Returns most fitting point near a given point in an image.
\begin{quote}\begin{description}
\item[{Parameters}] \leavevmode\begin{itemize}
\item {} 
\sphinxstyleliteralstrong{img} (\sphinxstyleliteralemphasis{ndarray}) -- Image in which point is searched.

\item {} 
\sphinxstyleliteralstrong{point} (\sphinxstyleliteralemphasis{tuple}) -- Point near which the best point/corner is searched.

\end{itemize}

\item[{Returns}] \leavevmode
\sphinxstylestrong{point} -- Best point.

\item[{Return type}] \leavevmode
tuple

\end{description}\end{quote}

\end{fulllineitems}

\index{getPointFromRectangle() (in module image\_sac)}

\begin{fulllineitems}
\phantomsection\label{\detokenize{image_sac:image_sac.getPointFromRectangle}}\pysiglinewithargsret{\sphinxcode{image\_sac.}\sphinxbfcode{getPointFromRectangle}}{\emph{img1}, \emph{point1}, \emph{point2}}{}
Computes point of interest in a rectangle defined by two given points.

The best corner is returned weighted by the distance to the center of the rectangle.
\begin{quote}\begin{description}
\item[{Parameters}] \leavevmode\begin{itemize}
\item {} 
\sphinxstyleliteralstrong{img1} (\sphinxstyleliteralemphasis{ndarray}) -- Image in which corner  point of interest is searched.

\item {} 
\sphinxstyleliteralstrong{point1} (\sphinxstyleliteralemphasis{tuple}) -- First corner of rectangle.

\item {} 
\sphinxstyleliteralstrong{point2} (\sphinxstyleliteralemphasis{tuple}) -- Opposite corner of first corner of rectangle.

\end{itemize}

\item[{Returns}] \leavevmode
\sphinxstylestrong{returnPoint} -- The best point of interest.

\item[{Return type}] \leavevmode
tuple

\end{description}\end{quote}

\end{fulllineitems}

\index{get\_and\_pre\_patch() (in module image\_sac)}

\begin{fulllineitems}
\phantomsection\label{\detokenize{image_sac:image_sac.get_and_pre_patch}}\pysiglinewithargsret{\sphinxcode{image\_sac.}\sphinxbfcode{get\_and\_pre\_patch}}{\emph{img}, \emph{point}, \emph{size\_half}}{}
Get a patch from an image and preprocess it.

Algorithm returns biggest possible template limited by the image size.
\begin{quote}\begin{description}
\item[{Parameters}] \leavevmode\begin{itemize}
\item {} 
\sphinxstyleliteralstrong{img} (\sphinxstyleliteralemphasis{ndarray}) -- Image from which to get the patch.

\item {} 
\sphinxstyleliteralstrong{point} (\sphinxstyleliteralemphasis{ndarray}) -- Centerpoint of the patch.

\item {} 
\sphinxstyleliteralstrong{size\_half} (\sphinxstyleliteralemphasis{int}) -- Half of the patchsize.

\end{itemize}

\item[{Returns}] \leavevmode
\begin{itemize}
\item {} 
\sphinxstylestrong{subimage} (\sphinxstyleemphasis{ndarray}) -- Template as copy.

\item {} 
\sphinxstyleemphasis{ndarray} -- Offset of lowest corner of patch with respect to image origin.

\item {} 
\sphinxstyleemphasis{ndarray} -- Differences between theorethical template size and practical
template size limited by size of the image.

\end{itemize}


\end{description}\end{quote}

\end{fulllineitems}



\chapter{ler package}
\label{\detokenize{ler::doc}}\label{\detokenize{ler:ler-package}}

\section{Submodules}
\label{\detokenize{ler:submodules}}

\section{ler.image\_ler module}
\label{\detokenize{ler:ler-image-ler-module}}\label{\detokenize{ler:module-ler.image_ler}}\index{ler.image\_ler (module)}
Find height, width of the largest rectangle containing all 0's in the matrix.

The algorithm for \sphinxtitleref{max\_size()} is suggested by @j\_random\_hacker {[}1{]}.
The algorithm for \sphinxtitleref{max\_rectangle\_size()} is from {[}2{]}.
The Python implementation {[}3{]} is dual licensed under CC BY-SA 3.0
and ISC license.

{[}1{]}: \url{http://stackoverflow.com/questions/2478447/find-largest-rectangle-containing-only-zeros-in-an-nn-binary-matrix\#comment5169734\_4671342}

{[}2{]}: \url{http://blog.csdn.net/arbuckle/archive/2006/05/06/710988.aspx}

{[}3{]}: \url{http://stackoverflow.com/a/4671342}

Copyright (c) 2014, zed \textless{}\href{mailto:isidore.john.r@gmail.com}{isidore.john.r@gmail.com}\textgreater{}

Permission to use, copy, modify, and/or distribute this software for any
purpose with or without fee is hereby granted, provided that the above
copyright notice and this permission notice appear in all copies.

THE SOFTWARE IS PROVIDED ``AS IS'' AND THE AUTHOR DISCLAIMS ALL WARRANTIES
WITH REGARD TO THIS SOFTWARE INCLUDING ALL IMPLIED WARRANTIES OF
MERCHANTABILITY AND FITNESS. IN NO EVENT SHALL THE AUTHOR BE LIABLE FOR
ANY SPECIAL, DIRECT, INDIRECT, OR CONSEQUENTIAL DAMAGES OR ANY DAMAGES
WHATSOEVER RESULTING FROM LOSS OF USE, DATA OR PROFITS, WHETHER IN AN
ACTION OF CONTRACT, NEGLIGENCE OR OTHER TORTIOUS ACTION, ARISING OUT OF
OR IN CONNECTION WITH THE USE OR PERFORMANCE OF THIS SOFTWARE.

Changed by \href{mailto:axschaffland@uos.de}{axschaffland@uos.de} to return not max\_size but position of
largest empty rectangle.
\index{Info (class in ler.image\_ler)}

\begin{fulllineitems}
\phantomsection\label{\detokenize{ler:ler.image_ler.Info}}\pysiglinewithargsret{\sphinxstrong{class }\sphinxcode{ler.image\_ler.}\sphinxbfcode{Info}}{\emph{start}, \emph{height}}{}
Bases: \sphinxcode{tuple}
\index{height (ler.image\_ler.Info attribute)}

\begin{fulllineitems}
\phantomsection\label{\detokenize{ler:ler.image_ler.Info.height}}\pysigline{\sphinxbfcode{height}}
Alias for field number 1

\end{fulllineitems}

\index{start (ler.image\_ler.Info attribute)}

\begin{fulllineitems}
\phantomsection\label{\detokenize{ler:ler.image_ler.Info.start}}\pysigline{\sphinxbfcode{start}}
Alias for field number 0

\end{fulllineitems}


\end{fulllineitems}

\index{TestCase (class in ler.image\_ler)}

\begin{fulllineitems}
\phantomsection\label{\detokenize{ler:ler.image_ler.TestCase}}\pysiglinewithargsret{\sphinxstrong{class }\sphinxcode{ler.image\_ler.}\sphinxbfcode{TestCase}}{\emph{methodName='runTest'}}{}
Bases: \sphinxcode{unittest.case.TestCase}
\index{test() (ler.image\_ler.TestCase method)}

\begin{fulllineitems}
\phantomsection\label{\detokenize{ler:ler.image_ler.TestCase.test}}\pysiglinewithargsret{\sphinxbfcode{test}}{}{}
\end{fulllineitems}


\end{fulllineitems}

\index{area() (in module ler.image\_ler)}

\begin{fulllineitems}
\phantomsection\label{\detokenize{ler:ler.image_ler.area}}\pysiglinewithargsret{\sphinxcode{ler.image\_ler.}\sphinxbfcode{area}}{\emph{size}}{}
\end{fulllineitems}

\index{max\_rectangle\_size() (in module ler.image\_ler)}

\begin{fulllineitems}
\phantomsection\label{\detokenize{ler:ler.image_ler.max_rectangle_size}}\pysiglinewithargsret{\sphinxcode{ler.image\_ler.}\sphinxbfcode{max\_rectangle\_size}}{\emph{histogram}}{}
Find height, width of the largest rectangle that fits entirely under
the histogram.

\begin{sphinxVerbatim}[commandchars=\\\{\}]
\PYG{g+gp}{\PYGZgt{}\PYGZgt{}\PYGZgt{} }\PYG{n}{f} \PYG{o}{=} \PYG{n}{max\PYGZus{}rectangle\PYGZus{}size}
\PYG{g+gp}{\PYGZgt{}\PYGZgt{}\PYGZgt{} }\PYG{n}{f}\PYG{p}{(}\PYG{p}{[}\PYG{l+m+mi}{5}\PYG{p}{,}\PYG{l+m+mi}{3}\PYG{p}{,}\PYG{l+m+mi}{1}\PYG{p}{]}\PYG{p}{)}
\PYG{g+go}{(3, 2)}
\PYG{g+gp}{\PYGZgt{}\PYGZgt{}\PYGZgt{} }\PYG{n}{f}\PYG{p}{(}\PYG{p}{[}\PYG{l+m+mi}{1}\PYG{p}{,}\PYG{l+m+mi}{3}\PYG{p}{,}\PYG{l+m+mi}{5}\PYG{p}{]}\PYG{p}{)}
\PYG{g+go}{(3, 2)}
\PYG{g+gp}{\PYGZgt{}\PYGZgt{}\PYGZgt{} }\PYG{n}{f}\PYG{p}{(}\PYG{p}{[}\PYG{l+m+mi}{3}\PYG{p}{,}\PYG{l+m+mi}{1}\PYG{p}{,}\PYG{l+m+mi}{5}\PYG{p}{]}\PYG{p}{)}
\PYG{g+go}{(5, 1)}
\PYG{g+gp}{\PYGZgt{}\PYGZgt{}\PYGZgt{} }\PYG{n}{f}\PYG{p}{(}\PYG{p}{[}\PYG{l+m+mi}{4}\PYG{p}{,}\PYG{l+m+mi}{8}\PYG{p}{,}\PYG{l+m+mi}{3}\PYG{p}{,}\PYG{l+m+mi}{2}\PYG{p}{,}\PYG{l+m+mi}{0}\PYG{p}{]}\PYG{p}{)}
\PYG{g+go}{(3, 3)}
\PYG{g+gp}{\PYGZgt{}\PYGZgt{}\PYGZgt{} }\PYG{n}{f}\PYG{p}{(}\PYG{p}{[}\PYG{l+m+mi}{4}\PYG{p}{,}\PYG{l+m+mi}{8}\PYG{p}{,}\PYG{l+m+mi}{3}\PYG{p}{,}\PYG{l+m+mi}{1}\PYG{p}{,}\PYG{l+m+mi}{1}\PYG{p}{,}\PYG{l+m+mi}{0}\PYG{p}{]}\PYG{p}{)}
\PYG{g+go}{(3, 3)}
\PYG{g+gp}{\PYGZgt{}\PYGZgt{}\PYGZgt{} }\PYG{n}{f}\PYG{p}{(}\PYG{p}{[}\PYG{l+m+mi}{1}\PYG{p}{,}\PYG{l+m+mi}{2}\PYG{p}{,}\PYG{l+m+mi}{1}\PYG{p}{]}\PYG{p}{)}
\PYG{g+go}{(1, 3)}
\end{sphinxVerbatim}

Algorithm is ``Linear search using a stack of incomplete subproblems'' {[}1{]}.

{[}1{]}: \url{http://blog.csdn.net/arbuckle/archive/2006/05/06/710988.aspx}

\end{fulllineitems}

\index{max\_size() (in module ler.image\_ler)}

\begin{fulllineitems}
\phantomsection\label{\detokenize{ler:ler.image_ler.max_size}}\pysiglinewithargsret{\sphinxcode{ler.image\_ler.}\sphinxbfcode{max\_size}}{\emph{mat}, \emph{value=0}}{}
Find height, width of the largest rectangle containing all \sphinxtitleref{value}`s.

For each row solve ``Largest Rectangle in a Histrogram'' problem {[}1{]}:

{[}1{]}: \url{http://blog.csdn.net/arbuckle/archive/2006/05/06/710988.aspx}

\end{fulllineitems}



\section{Module contents}
\label{\detokenize{ler:module-contents}}\label{\detokenize{ler:module-ler}}\index{ler (module)}

\chapter{spqr package}
\label{\detokenize{spqr::doc}}\label{\detokenize{spqr:spqr-package}}

\section{Submodules}
\label{\detokenize{spqr:submodules}}

\section{spqr.image\_spqr module}
\label{\detokenize{spqr:spqr-image-spqr-module}}\label{\detokenize{spqr:module-spqr.image_spqr}}\index{spqr.image\_spqr (module)}\index{main() (in module spqr.image\_spqr)}

\begin{fulllineitems}
\phantomsection\label{\detokenize{spqr:spqr.image_spqr.main}}\pysiglinewithargsret{\sphinxcode{spqr.image\_spqr.}\sphinxbfcode{main}}{}{}
\end{fulllineitems}

\index{qr\_solve() (in module spqr.image\_spqr)}

\begin{fulllineitems}
\phantomsection\label{\detokenize{spqr:spqr.image_spqr.qr_solve}}\pysiglinewithargsret{\sphinxcode{spqr.image\_spqr.}\sphinxbfcode{qr\_solve}}{\emph{A\_data}, \emph{A\_row}, \emph{A\_col}, \emph{A\_nnz}, \emph{A\_m}, \emph{A\_n}, \emph{b\_data}}{}
Python wrapper to qr\_solve.\textless{}

\end{fulllineitems}



\section{Module contents}
\label{\detokenize{spqr:module-contents}}\label{\detokenize{spqr:module-spqr}}\index{spqr (module)}

\chapter{Indices and tables}
\label{\detokenize{index:indices-and-tables}}\begin{itemize}
\item {} 
\DUrole{xref,std,std-ref}{genindex}

\item {} 
\DUrole{xref,std,std-ref}{modindex}

\item {} 
\DUrole{xref,std,std-ref}{search}

\end{itemize}


\renewcommand{\indexname}{Python Module Index}
\begin{sphinxtheindex}
\def\bigletter#1{{\Large\sffamily#1}\nopagebreak\vspace{1mm}}
\bigletter{i}
\item {\sphinxstyleindexentry{image\_aaap}}\sphinxstyleindexpageref{image_aaap:\detokenize{module-image_aaap}}
\item {\sphinxstyleindexentry{image\_aaap\_main}}\sphinxstyleindexpageref{image_aaap_main:\detokenize{module-image_aaap_main}}
\item {\sphinxstyleindexentry{image\_draw\_grid}}\sphinxstyleindexpageref{image_draw_grid:\detokenize{module-image_draw_grid}}
\item {\sphinxstyleindexentry{image\_gabor}}\sphinxstyleindexpageref{image_gabor:\detokenize{module-image_gabor}}
\item {\sphinxstyleindexentry{image\_helpers}}\sphinxstyleindexpageref{image_helpers:\detokenize{module-image_helpers}}
\item {\sphinxstyleindexentry{image\_io}}\sphinxstyleindexpageref{image_io:\detokenize{module-image_io}}
\item {\sphinxstyleindexentry{image\_lines}}\sphinxstyleindexpageref{image_lines:\detokenize{module-image_lines}}
\item {\sphinxstyleindexentry{image\_morphing}}\sphinxstyleindexpageref{image_morphing:\detokenize{module-image_morphing}}
\item {\sphinxstyleindexentry{image\_perspective\_alignment}}\sphinxstyleindexpageref{image_perspective_alignment:\detokenize{module-image_perspective_alignment}}
\item {\sphinxstyleindexentry{image\_sac}}\sphinxstyleindexpageref{image_sac:\detokenize{module-image_sac}}
\indexspace
\bigletter{l}
\item {\sphinxstyleindexentry{ler}}\sphinxstyleindexpageref{ler:\detokenize{module-ler}}
\item {\sphinxstyleindexentry{ler.image\_ler}}\sphinxstyleindexpageref{ler:\detokenize{module-ler.image_ler}}
\indexspace
\bigletter{s}
\item {\sphinxstyleindexentry{spqr}}\sphinxstyleindexpageref{spqr:\detokenize{module-spqr}}
\item {\sphinxstyleindexentry{spqr.image\_spqr}}\sphinxstyleindexpageref{spqr:\detokenize{module-spqr.image_spqr}}
\end{sphinxtheindex}

\renewcommand{\indexname}{Index}
\printindex
\end{document}